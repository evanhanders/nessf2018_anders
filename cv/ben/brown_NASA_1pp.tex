\documentclass[11pt, preprint]{aastex}

%%%%%%begin preamble
\usepackage[hmargin=1in, vmargin=1in]{geometry} % Margins
\usepackage{hyperref}
\usepackage{url}
\usepackage{times}
\usepackage{natbib}
\usepackage{graphicx}
\usepackage{amsmath}
\usepackage{amsfonts}
\usepackage{amssymb}
\usepackage{pdfpages}
\usepackage{import}
%\usepackage{fontspec}
%\setmainfont{TimesNewRoman}
%\definetypeface[Serapion][rm][Xserif][Serapion Pro]
%\setupbodyfont[Serapion, 12pt]

%%%
%%%%%% uncomment following 4 lines to adjust title size/shape and
%%%%%% trailing space
%% \usepackage{titling}
%% %\pretitle{\noindent\Large\bfseries}
%% \date{}
%% \setlength{\droptitle}{-1in}
%\posttitle{\\}

\hypersetup{
     colorlinks   = true,
     citecolor    = gray,
     urlcolor      = blue
}

\setcounter{tocdepth}{2}
%% headers
\usepackage{fancyhdr}
\pagestyle{fancy}
\lhead{Brown}
\chead{}
\rhead{}
\lfoot{}
\cfoot{\thepage}
\rfoot{}

\newcommand{\sol}{\ensuremath{\odot}}
\newcommand{\Dedalus}{\href{http://dedalus-project.org}{Dedalus}}
\newcommand{\Reyn}{\ensuremath{\mathrm{Re}}}
\newcommand{\Rayleigh}{\ensuremath{\mathrm{Ra}}}
\newcommand{\Rossby}{\ensuremath{\mathrm{Ro}}}
\newcommand{\Rmag}{\ensuremath{\mathrm{Rm}}}
\newcommand{\Rmagc}{\ensuremath{\mathrm{Rm}_\mathrm{crit}}}
\newcommand{\Prandtl}{\ensuremath{\mathrm{Pm}}}
\newcommand{\Peclet}{\ensuremath{\mathrm{Pe}}}
\newcommand{\Mach}{\ensuremath{\mathrm{Ma}}}
\newcommand{\Stiffness}{\ensuremath{\mathrm{S}}}
\newcommand{\Lund}{\ensuremath{\mathrm{S}}}
\newcommand{\Lundc}{\ensuremath{\mathrm{S}_\mathrm{crit}}}
\newcommand{\yt}{\texttt{yt}}
\newcommand{\enzo}{\texttt{Enzo}}
\newcommand{\nosection}[1]{%
  \refstepcounter{section}%
  \addcontentsline{toc}{section}{\protect\numberline{\thesection}#1}%
  \markright{#1}}
\newcommand{\nosubsection}[1]{%
  \refstepcounter{subsection}%
  \addcontentsline{toc}{subsection}{\protect\numberline{\thesubsection}#1}%
  \markright{#1}}


%%%%%%end preamble

\begin{document}
\thispagestyle{empty}
\parindent=0cm
\section*{~}
\vspace{-1.5cm}
\begin{center}
\huge{Benjamin Brown}\\
\large{Biographical Sketch}\\
\small{Department of Astrophysical and Planetary Sciences\\
University of Colorado, Boulder \\\texttt{bpbrown@colorado.edu}}
\end{center}

\vspace{-1.25cm}
\section*{Professional Preparation}
\vspace{-0.25cm}
\begin{tabular}{lll}
  Harvey Mudd College & Physics & BS May 2003\\
  University of Colorado, Boulder & Astrophysics & PhD August 2009\\
  University of Wisconsin, Madison & Astronomy & Postdoc September 2009 -- August 2013 \\
  University of California, Santa Barbara & KITP  & Postdoc September 2013 -- July 2014\\
\end{tabular}
\vspace{-0.75cm}
\section*{Appointments}
\vspace{-0.25cm}
\begin{tabular}{lll}
  Assistant Professor & University of Colorado, Boulder & August 2014 -- \\
  Research Associate & Kavli Institute for Theoretical Physics & September 2013 -- July 2014\\
  Postdoctoral Fellow & University of Wisconsin, Madison & September 2009 -- August 2013\\
  NSF AAPF  & University of Wisconsin, Madison & September 2009 -- August 2013\\
\end{tabular}
\vspace{-0.75cm}
\section*{Five Publications Most Relevant to Proposed Work (out of 34 total)}
\vspace{-0.25cm}
\begin{itemize}
\setlength{\itemsep}{-\parsep}
\setlength{\topsep}{-2\parsep}
\setlength{\partopsep}{-2\parsep}

\item Bordwell, B., \textbf{Brown}, B.~P., \& Oishi, J.~S.,
  ``Convective dynamics and disequilibrium chemistry in the atmospheres of giant planets and brown dwarfs'',
  2018, \emph{The Astrophysical Journal}, in press


\item Lecoanet, D., Schwab, J., Quataert, E., Bildsten, L., Timmes,
  F.~X., Burns, K.~J., Vasil, G.~M., Oishi, J.~S., \& \textbf{Brown}, B.~P.,
  ``Turbulent chemical diffusion in convectively bounded carbon  flames'',
  2016, \emph{The Astrophysical Journal}, 832,~71:1--8

\item  Lecoanet, D., \textbf{Brown}, B.~P., Zweibel, E.~G., Burns, K., Oishi, J.~S, \& Vasil, G.~M.,
``Conduction in low-Mach number flows: part I linear \& weakly nonlinear regimes'',
2014, \emph{The Astrophysical Journal}, 797,~94:1--16

\item
  Vasil, G.~M., Lecoanet, D., \textbf{Brown}, B.~P., Wood, T.~S., \& {Zweibel}, E.~G.,
``Energy conservation and gravity waves in sound-proof treatments of
  stellar interiors: Part II Lagrangian constrained analysis'', 2013,
  \textit{The Astrophysical Journal} 773,~169:1--23

\item
\textbf{Brown}, B.~P.,  Vasil, G.~M., \& Zweibel, E.~G., 
``Energy conservation and gravity waves in sound-proof treatments of
  stellar interiors: Part I anelastic approximations'', 2012,
  \textit{The Astrophysical Journal} 756,~109:1--20

\end{itemize}

\vspace{-1.5cm}
\section*{Synergistic activities}
\vspace{-0.25cm}
Brown is an expert in the stratified fluid dynamics of stars and planetary atmospheres.
Brown has been involved in modelling stellar convection since 2003,
when he began using the anelastic spherical harmonic (ASH) code to
study the coupling of convection, rotation and magnetic dynamo action
in the Sun and in other solar-type stars.  He has published results on magnetohydrodynamic
processes in stellar interiors, on convective wave generation and
transport, and on fundamental properties of stratified fluid dynamics.
He is a core member of the development team for the open-source
Dedalus framework.  He has extensive HPC experience and a history of 
success in obtaining large computing allocations.  
At University of Colorado, he leads a research
group of four graduate students, two postdocs and two undergraduate
students, working on topics in solar, stellar and exoplanetary
dynamics.  He has mentored two students through Masters level
(Anders \& Bordwell); both recieved highest honors in their research.

\end{document}