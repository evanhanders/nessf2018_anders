\documentclass[aasms,12pt]{article}
\usepackage{natbib}
\setlength{\bibsep}{0pt plus 0.3ex}
\usepackage[margin=0.7in]{geometry}
\usepackage{sectsty}
\usepackage{graphicx}
\usepackage{hyperref}
\usepackage{epstopdf}
\usepackage[skip=2pt,font=small]{caption}
\captionsetup{width=\textwidth}
\usepackage{amssymb, amsmath, amsfonts, xcolor}
\hypersetup{
    colorlinks,
    linkcolor={red!50!black},
    citecolor={blue!80!black},
    urlcolor={blue!80!black}
}


%\usepackage{titlesec}
%
%\titlespacing\section{0pt}{12pt plus 4pt minus 2pt}{5pt plus 2pt minus 2pt}
%\titlespacing\subsection{0pt}{12pt plus 4pt minus 2pt}{0pt plus 2pt minus 2pt}
%\titlespacing\subsubsection{0pt}{12pt plus 4pt minus 2pt}{0pt plus 2pt minus 2pt}


\sectionfont{\normalsize}
\subsectionfont{\small}
%\usepackage{fullpage}


%\citestyle{aa}
\newcommand{\sol}{\ensuremath{\odot}}
\newcommand{\RB}{Rayleigh-B\'{e}nard }
\newcommand{\grad}{\ensuremath{\nabla}}

\usepackage{fancyhdr}
\pagestyle{fancy}
\fancyhf{} % sets both header and footer to nothing
\renewcommand{\headrulewidth}{0pt}





\begin{document}
\begin{center}
   \large\textbf{Towards a more complete understanding of solar convection}\\
   \vspace{0.4cm}
   \large{Evan H. Anders}\\
   \vspace{0.4cm}
   \normalsize\textit{Advisor: Benjamin P. Brown}\\
   \normalsize\textit{Laboratory for Atmospheric and Space Physics (LASP) \& University of Colorado at Boulder}\\
\end{center}

\section{Background \& Motivation}
The Sun exhibits active magnetism which cycles in magnitude every 11 years.
This magnetism arises from an 
organized dynamo seated in the turbulent plasma
motions of the solar convective zone, which occupies roughly the outer 30\%
of the Sun's radius. This activity manifests itself in the collection of phenomena generally
referred to as solar activity, including magnetic storms and coronal mass
ejections.  Such activity propagates towards Earth, threatening disruption of 
power grids and aircraft operations as well as endangering astronauts and satellites.
The motivation to understand the Sun's magnetism in our increasingly technological society
is great.  A critical step in protecting our society from the Sun's magnetism is 
to gain an understanding of the nature of the dynamo that generates the Sun's magnetic fields
\cite{nordlund&all2009, charbonneau2014}.

To understand the dynamo that drives the Sun's magnetism, we must understand the convection
which powers that dynamo.  Numerical studies of convection in stratified domain have a
rich history in the past decade.  The early work of \cite{graham1975}, \cite{hurlburt&all1984},
and others in simple, plane-parallel atmospheres
provided rich insight into the nature of stratified convection and provided a basis
in a field which now regularly creates both complex, 3D global models of convection
(e.g., \cite{brown&all2010} and \cite{guerrero&all2016})
and smaller scale local area models with more complex physics 
(e.g., \cite{stein&nordlund2012} and \cite{rempel2014}).
From these efforts we have learned a great deal about the nature of convection, and have
even created beautiful simulations which even \emph{look} like the convection we see on the
surface of the Sun.

Unfortunately, the great advances made in computational prowess within the solar physics
community seem to have surpassed our fundamental knowledge in the field. There is currently
a so-called ``Solar Convective Conundrum'' which has two components.  Both components of
this conundrum are present in the recent observations
by \cite{hanasoge&all2012} (Fig. \ref{fig:fig1}a).  First, they observed solar convective velocities 
two orders of magnitude smaller than theory predicts. Second, their observations showed that there
was less power at large length scales than short length scales -- exactly the opposite of what
we expect.
The two-part convective conundrum -- the presence of low convective amplitudes
and the lack of ``giant cells'' at large length scales -- has baffled the community since it
came to light.

More recent work by \cite{greer&all2015} (Fig. \ref{fig:fig1}b) argue that the convective
velocity amplitude is perhaps not so low as previously reported, but there is still a distinct
lack of giant cells imprinting on the near-surface flows in this work.
Even simpler doppler measurements of the velocity fields at the solar surface which are not muddied
by complex helioseismic inversions lack
giant cells (e.g., \cite{hathaway&all2015} \& Fig. \ref{fig:fig1}c).  
The motions of surface granules and the slightly deeper supergranules are clearly present, but no
larger length scale is distinct.

The difficulty in the lack of giant cells is exemplified by the work of \cite{lord&all2014}, who showed that 
the length scale of convective motions is determined by the depth in the atmosphere at which they are driven.
Unstable, convecting layers drive motions whose scale is proportionate to
the local density scale height.  Deep in the atmosphere, the density scale height is large, and so we expect
the bottom of the convection zone to drive large, giant cells which imprint through to the surface.
The lack of their presence shows either a lack of our fundamental understanding of convection or
a systematic problem with our observational methods -- and the former is more likely.

Our inability to observe giant cells hints at the possibility that giant cells are never driven in the first place,
which would mean that the deep convection zone is not stably stratified. Some recent work
(\cite{brandenburg2016}, \cite{kapyla&all2017}) has sought to understand the nature of convection
-- and explain the lack of giant cells -- when nonlocal effects are included and observed.  
These efforts are, however, often extensions of already complex theory (Mixing Length Theory) or
simulations which include a large amount of physics, but not a careful study of how those physics
determine the solution.

Drawing on the knowledge and expertise of those in the physics community who study incompressible,
unstratified \RB convection, I have recently examined hydrodynamic, compressible convection
in simple stratified domains \cite{anders&brown2017}.  I discovered that these somewhat complicated systems
\emph{transport heat largely in the same way as simple incompressible systems}.  
This was accomplished by setting up a simple experiment, understanding the different control parameters
on the experiment and how they affected the convective states, and then setting up a controlled parameter
space study.

Here I present two simple, small experiments which aim to understand the nature of convection
in which giant cells are not driven.  I will use methods similar to that in my recently published work
to create simple, controlled experiments on comprehensible atmospheres in order to gain a deep
understanding of the underlying convective physics.

\begin{figure}[t]
\centering
\includegraphics[width=\textwidth]{figs/fig1.png}
\caption{(a) Power spectra of solar convective velocities are shown for both observations and simulations,
and both near the surface and deeper \cite{hanasoge&all2012}.  Observations are obtained
using time-distance helioseismology, and show velocities roughly two orders of magnitude lower
than those predicted by simulations, and a decrease in power approaching larger length
scales rather than the opposite. (b) Further observations of solar velocity power  using
ring-diagram helioseismology.  Here, velocity magnitudes are roughly in line with those
predicted by simulations, but show decreasing power as larger scales are approached, unlike
what is expected from simulations \cite{greer&all2015}.  A simple spectrum of horizontal
velocities at the solar surface, obtained using line-of-sight Doppler velocities \cite{hathaway&all2015}.
The length scales of surface granules and deeper supergranules appear as distinct features, but
the hypothesized giant cells are not observed at low wavenumber.
        \label{fig:fig1}}
\end{figure}

\section{Proposed Project}
In simple simulations of convection, motions are often driven by enforced boundary conditions
on the thermodynamic state.  This creates boundary layers at the top and bottom of the atmosphere,
and convection is strongly driven within those boundary layers.  The situation in the Sun is less
black-and-white.  There is a positive radial gradient of opacity within the Sun, such that at
radii near the solar surface, radiative flux is inefficient and convection is required to carry
the solar luminosity outwards.  This divergence of radiative flux acts like an internal heating term
which drives the convection weakly at the base of the solar convective zone and more strongly near the
top of the solar convection zone.  Further, the upper boundary layer of the Sun is not determined by the
local thermal diffusivity (which is very small), but rather by the ionization and recombination of hydrogen.

We propose two simple experiments to test these two aspects of solar convective driving.  In the first,
we will study the nature of Kramers' opacity, which decreases with depth, and the internally heated
convection which this drives.  In the second, we will carefully study the effects of hydrogen ionization
on convection in order to determine the nature of convection when new physics determines a different length
scale for the driving region than the natural physics of the thermal boundary layer.

\subsection{Mini Project 1: Kramers' Opacity}
Many careful studies of convection employ a constant opacity and thus, a constant conductivity.
The transport of heat within the atmosphere, in addition to convective transport, is often 
quantified through Fourier's law of conduction \citep{lecoanet&all2014}, in which the radiative
(conductive) flux is proportional to the conductivity and the temperature gradient.  
A constant opacity in time and space allows for the creation of simple measurements of the
heat transport in the evolved atmosphere compared to the initial atmosphere.  

While a constant
conductivity is the go-to choice for many in the physics community who study incompressible
\RB convection, it is often not the choice for those in the heliophysics or astrophysics
communities.  Instead, these communities generally employ a Kramers' opacity, in which the
opacity is proportional to $T^3/\rho$, where $T$ is temperature and $\rho$ is density.
This means that the conductivity is proportional to $\rho/T^3$, which changes significantly
throughout the depth of the solar convection zone.  Unfortunately, there have never been
systematic, careful studies of the effects of Kramers' opacity on stratified convection.

What experimental knob determines the Mach number?  At what value of the Rayleigh number
does convection turn on (and thus, at what \emph{supercriticality} are other studies
being run)?  What is the appropriate reference state which is in hydrostatic and thermal
equilibrium?  What are the parameters of that reference state that determine key quantities
of the evolved convection, and what can we learn about the evolved convection from them?

After determining what the appropriate reference state is for a simple convective experiment
where the Kramer's opacity is operating, and after determining how to systematically compare
similar atmospheres at low and high Mach number, we will study the importance of nonlinearities
in this opacity term on the resultant convection.  In downflows (where density is high
and temperature is low), we expect the opacity to be small compared to downflows, and
the opposite to be true for upflows.  If this is the case, then the ``entropy rain'' formulation
presented in \cite{brandenburg2016} may be an appropriate adjustment to the Mixing Length
theory of convection.  However, at low Mach number, where variations in $T$ and $\rho$ are small
in upflows compared to downflows, we anticipate that this effect will be unimportant.  Near the
solar surface, the Mach number is nearly 1, and this is likely significant.  However, deep in
the convective zone, the Mach number is very small (O($10^{-5}$)), and so this effect is
likely quite unimportant.

We want to quantify \emph{how important} the nonlinearities in the opacity are for 
nonlocal transport, and also what effects these might have on the resultant stratification
(and thus convective driving) of the solar convection zone.


%Our first experiment involves the study of atmospheres where a constant internal heating
%term is included in the energy equation.  This is the simplest form of internal heating,
%and is similar to that which has been studied in \RB convection, just with compressibility
%and stratification included (e.g., \cite{goluskin&spiegel2012}).  
%We have found that these systems can be constructed so as to
%naturally include stable layers underlying unstable, convection regions.  These systems
%are important to study, because the changing opacities in the interior of the solar convective
%zone effectively act like an internal heating term, depositing energy and driving convection.
%
%We find that these systems essentially have the same input control parameters as simple polytropic
%convection \cite{anders&brown2017}, which include the Rayleigh number (Ra, the strength of
%convective driving and amount of turbulence), the Prandtl number (Pr, the ratio of the thermal
%to viscous diffusivity), the characteristic superadiabaticity of the atmosphere (which sets the Mach
%number of the flows), and the depth of the atmosphere.  In the internally heated systems, we
%find that the magnitude of the internal heating directly relates to the superadiabaticity of the
%atmosphere, and thus Ma, but we must split up the depth into two parameters: the depth of the
%\emph{unstable} region, and the depth of the new \emph{stable} region.  In \cite{anders&brown2017},
%we explored the effects of the magnitude of the superadiabaticity on the resulting convection,
%and found that -- except where the Mach number is very high -- it has almost no effect on the
%other properties of the convective flows.  
%
%In this project, we will focus on understanding how the depth of the atmosphere, and the amount
%of stratification felt by the flows, affects the resultant convection.  We will set the depth
%of the convection zone (to 1, 3, and then 5 density scale heights), and for each of these depths
%we will run simulations for shallow and deep radiative zones.  We will examine the evolved
%stratification of these atmospheres in order to determine where convection is driven and where
%it is not.  Preliminary results show that the so-called ``Deardorff zones'' present in the
%simulations of \cite{kapyla&all2017} are also present in these simple simulations of
%internally heated convection and do not require complex forms of the conductivity to be achieved.
%We expect that the extent of these subadiabatic zones in which convection carries the flux can
%be determined \emph{a priori} in these simple simulations, but further work must be done.
%
%We're looking for the motions at the surface and also the stratification.
%
%
%Our research group has extensive experience studying convection penetrating into stable layers.
%In soon-to-be-submitted work (Brown et al. 2018, in prep) a systematic approach to studying
%penetrative convection is laid out, and it is found that the depth of penetration is well
%predicted by a buoyancy equilibration model.  We anticipate that the same will be true in 
%Internaly heated convection.  Another piece of forthcoming work (Oishi et al. 2018, in prep)
%shows that penetrating convection does not extend the convection zone downwards into the stable
%layer, as typical knowledge assumes.  It rather dredges up excess entropy into the convecting region
%and makes the bottom of the convecting region marginally stable.  A similar effect was found
%recently in more complex systems \cite{kapyla&all2017}, and is also observed in our preliminary
%runs of internally heated convection.  This work will study this phenomenon carefully and apply
%this knowledge to the Solar Convective Conundrum.

\subsection{Mini Project 2: Hydrogen recombination}
Most studies of convection are driven either by the choice of fixed boundary conditions
(hot at bottom, cold at top) or through the volumetric depoisition of energy throughout
the domain (internal heating).  Here we propose a project in which we utilize a nonlinear
equation of state to represent the ionization / reionization of hydrogen near the surface of
the Sun.  We examine the effects that this has on the driving boundary layer region near the
top of the simulation, and we do blahblahblah.

In order to carefully study the effects of the ionization of hydrogen, or other similar phase
changes, we must first understand the reference state for such an atmosphere.  It will no longer
simply be polytropic, as the nonlinear EOS rules out the simple polytropic assumption. Once the
appropriate state which is in hydrostatic and thermal equilibrium despite this phase change is discovered,
we will then carefully study the effects of the temperature of ionization (which will in turn
determine its depth), and also the ionization energy of the transition.  In standard simulations
of convection, the interactions of the flows with a hard boundary forms a thin boundary layer,
which scales downwards as the diffusivities shrink, and this boundary generally drives convection.
We suspect that we can create atmospheres with larger or smaller boundary layers than the natural
thermal boundary length scale, which will nominally drive convection on different length scales
than the natural one.  We are interested in seeing how these ``too large'' convective flows
interact with their surroundings, and in seeing how the atmosphere naturally evolves in the presence
of these flows.

We are particularly interested in determining the effects of \emph{where} in the atmosphere the
transition from ionized to neutral Hydrogen occurs, and also \emph{how much energy} is involved in
the ionization process.  These are two simple controls which can be examined in full detail through
a suite of simulations.  We hypothesize that, for sufficiently energetic ionization processes (such as
that of neutral hydrogen), a natural boundary layer will form between an overlying stable layer and
underlying convecting region in the atmosphere.  We are further interested in determining how the
ionization energy determines the length scale of the boundary layer.

Using these two control knobs, we are very interested in determining the startification of the evolved
solution.  This will tell us how large of a region is driving convection (is it a small layer 
near the reionization, or does it extend to a great depth below that?  We are also very interested
in the filling factor of convection compared to simple boundary-driven convection.


%\begin{figure}[t!]
%\centering
%%\includegraphics[width=14cm]{figs/2014_oct_sunspots.jpg}
%\caption{(a) word (b) word (c) word
%	\label{fig:fig2}}
%\end{figure}

\section{Numerical Tools and Feasibility}
I will use the open-source Dedalus\footnote{\url{http://dedalus-project.org/}} pseudospectral framework 
\cite{burns&all2016} to carry out my simulations.  
Dedalus is a flexible solver of general partial differential equations,
making it extremely easy to study diverse sets of equations under many different atmospheric
constraints.  I have already published one paper using this tool \cite{anders&brown2017},
have submitted another paper, and am now adept at using it to create suites of simulations
in short timeframes.

I will primarily study 2D convective solutions in plane-parallel atmospheres in order to gain
intuition about the mean behavior of vertical profiles within the atmosphere.  Once I have a grasp
on how my measurements vary in 2D across parameter space, I will run select 3D simulations to
verify whether or not that behavior holds in 3D, as I did in my previous paper \cite{anders&brown2017}.
In cases where 2D and 3D diverge, I will quantify how and why they do so, but most questions I am
asking are quite basic, and most of the systems I propose to study here have not been studied
in the compressible context, at least not recently.  By primarily studying in 2D, and by carefully
selecting my 3D runs once I know which parameters I must examine more carefully, I can complete
a full suite of simulations, such as those in my previous paper \cite{anders&brown2017}, using
roughly 3 million CPU-hours.  Through my advisor, I have access to an allocation on NASA Pleiades
of roughly 20 million CPU-hours/year, so one- or two- of the following projects of the scope I am
proposing can be completed each year.

Furthermore, my recent work (Anders, Brown, \& Oishi 2018, submitted to PRFluids) has
shown that properly constructed boundary value problems, coupled with initial value problems,
can fast-forward the slow thermal evolution of these convective simulations.  This work was done in 
Boussinesq, \RB convection but can be easily extended to stratified convection, and will greatly
extend both the number of simulations we are able to complete and the level of turbulent driving
(while attaining converged atmospheres) that we are able to solve.



\subsection{Timeline of proposed work}
\textbf{Year 1 (Fall 2018 - Summer 2019):}
\begin{itemize}
\vspace{-0.2cm}
\item Finalize work on Internally Heated convection, which will be started Spring-Summer 2018.
Submit a short paper summarizing the methods and results of this work to the Astrophysical Journal
by mid-Fall 2018.  Release code upon submission such that the community can use it.
\vspace{-0.2cm}
\item Delve into literature on past work regarding ionizing convection and moist convection.
Understand the pieces of physics necessary to correctly implement a nonlinear equation of state.
Begin to develop Dedalus simulations of ionizing convection by end of 2018.
\vspace{-0.2cm}
\item Fully Develop ionizing convection code.  Determine the range of parameter space to be studied
and execute the simulations within this range.  Analyze data, and have a short paper written and
submitted on them to the Astrophysical Journal by the end of summer 2019.
\end{itemize}

\textbf{Year 2 (Fall 2019 - Spring 2020):}
\begin{itemize}
\vspace{-0.2cm}
\item Begin literature review on Kramer's opacity late summer 2019-early fall 2020.  Understand
past work done, and implementing both time-dependent (fully nonlinear) and time-independent
versions of the Kramer's opacity in simple atmospheres.  Run a suite of simulations at high-
and low- Mach number by end of year 2019.
\vspace{-0.2cm}
\item Analyze data from Kramer's opacity results and prepare a small paper to submit to the
Astrophysical Journal Letters by end of Winter 2020.
\vspace{-0.2cm}
\item Combine work from five published papers into a thesis, to be defended at the end of 
Spring 2020.
\end{itemize}

\section{Relevance to NASA} 
The proposed work fits with NASA's 2014 Strategic Plan objective
1.4:
``Understand the Sun and its interactions with Earth and the solar
system, including space weather.''  Specifically, I aim to help answer
the fundamental question, ``What causes the Sun to vary?'' 
This work also aims to answer one of the three overarching science goals
in chapter 4.1 of NASA's 2014 Science Plan: 
``Develop the
knowledge and capability to detect and predict extreme conditions in space to
protect life and society and to safeguard human and robotic explorers beyond
earth.'' In order to understand how to predict space weather appropriately, we
need to understand the processes that cause this weather.  It is clear from
recent work that our understanding of the fundamentals of convection is not as perfect
as we once thought, and now is an exciting time to clarify our theory and determine which
parts of it fail and which parts hold true under more examination.  Only once we understand
the fundamental nature of stratified, compresible convection can we begin to understand
how it drives the dynamo in the Sun in the presence of many complications such as
differential rotation, shear layers near the base and top of the convection zone, and
magnetism.

The work has been motivated by data from the Helioseismic and Magnetic Imager (HMI) onboard
the NASA Solar Dynamics Observtory (SDO) spacecraft 
\cite{hanasoge&all2012, greer&all2015, hathaway&all2015}, and will continue to be informed by
new helioiseismic measurements made from SDO data, and from the new measurements which will
be made possible by the upcoming joint NASA-ESA Solar Orbiter's Polarimetric and 
Helioseismic Imager (PHI).


\section{Summary}
Recent observations call into question our fundamental understanding of stratified
convection in systems such as the solar convection zone \cite{hanasoge&all2012, greer&all2015}.
Here we present three focused, scoped studies of stratified convection which probe the
specific effects of individual elements of convection in the Sun.  Convection must carry
flux in the Sun's convective envelope, because the radiative flux becomes too small and deposits
energy there, acting essentially like internal heating. Convection is driven at the surface of
the Sun due to the ionization and recombination of hydrogen near the solar photosphere.  The
magnitude of the flux that must be carried by convection varies greatly throughout the depth
of the convection zone due to the increase of opacity with height.

We propose to probe the first of these elements by studying simple internal heating systems,
the second of these elements by studying convection with a nonlinear equation of state / phase
change, and the third of these elements by studying convection with a realistic opacity profile
in the context of our knowledge from the first two experiments.  Due to the developed nature of
our computational tool, Dedalus, the simulations for these projects can be implemented and
carried out on short timescales, and the body of work suggested here should be finished within
two years, by the end of the spring of 2019.


\newpage
\bibliographystyle{apj}
\bibliography{biblio}
\end{document}
