\documentclass[aasms,12pt]{article}
\usepackage{natbib}
\setlength{\bibsep}{0pt plus 0.3ex}
\usepackage[margin=1in]{geometry}
\usepackage{sectsty}
\usepackage{graphicx}
\usepackage{epstopdf}
\usepackage[skip=2pt,font=small]{caption}
\captionsetup{width=\textwidth}


%\usepackage{titlesec}
%
%\titlespacing\section{0pt}{12pt plus 4pt minus 2pt}{5pt plus 2pt minus 2pt}
%\titlespacing\subsection{0pt}{12pt plus 4pt minus 2pt}{0pt plus 2pt minus 2pt}
%\titlespacing\subsubsection{0pt}{12pt plus 4pt minus 2pt}{0pt plus 2pt minus 2pt}


\sectionfont{\normalsize}
%\usepackage{fullpage}


%\citestyle{aa}
\newcommand{\sol}{\ensuremath{\odot}}
\newcommand{\RB}{Rayleigh-B\'{e}nard }


\usepackage{fancyhdr}
\pagestyle{fancy}
\fancyhf{} % sets both header and footer to nothing
\renewcommand{\headrulewidth}{0pt}
\cfoot{\thepage}
\rfoot{\footnotesize{Evan H. Anders, NASA NESSF 2018}}




\begin{document}
\begin{center}
   \large\textbf{Towards a more complete understanding of Stratified, Compressible Convection}\\
   \vspace{0.4cm}
   \large{Evan H. Anders}\\
   \vspace{0.4cm}
   \normalsize\textit{Advisor: Benjamin Brown}\\
   \normalsize\textit{LASP, University of Colorado at Boulder}\\
\end{center}



\abstract{Blah blah blah. This project supports objective 1.4 of NASA's 2014 Strategic Plan
	and will assist in developing ``the
	knowledge and capability to detect and predict extreme conditions in
	space'' in accordance with NASA's overarching Heliophysics science goals.}



\section{Introduction}
The Sun exhibits active magnetism which cycles in magnitude every 11 years.
This magnetism arises from an 
organized dynamo seated in the turbulent plasma
motions of the solar convective zone, which occupies roughly the outer 30\%
of the Sun's radius. This activity manifests itself in the collection of phenomena generally
referred to as solar activity, including magnetic storms and coronal mass
ejections.  Such activity propagates towards Earth, threatening disruption of 
power grids and aircraft operations as well as endangering astronauts and satellites.
The motivation to understand the Sun's magnetism in our increasingly technological society
is great.  A critical step in protecting our society from the Sun's magnetism is 
to gain an understanding of the nature of the dynamo that generates the Sun's magnetic fields
\citep{nordlund&all2009, charbonneau2014}.

To understand the dynamo that drives the Sun's magnetism, we must understand the convection
which powers that dynamo.  Numerical studies of convection in stratified domain have a
rich history in the past decade.  The early work of \citealt{graham1975} and \citealt{hurlburt&all1984} and others in simple
domains provided rich insight into the nature of stratified convection and provided a basis
in a field which now regularly creates both complex, 3D global models of convection
(e.g., \citealt{brown&all2010}, \citealt{guerrero&all2016}, and many others)
and smaller scale local area models with more complex physics 
(e.g., \citealt{stein&nordlund2012}, \citealt{rempel2014}).
From these efforts we have learned a great deal about the nature of convection, and have
even created beautiful simulations which even \emph{look} like the convection we see on the
surface of the Sun.

Unfortunately, the great advances made in computational prowess within the solar physics
community seem to have surpassed our fundamental knowledge in the field. Recent observations
by \citealt{hanasoge&all2012} (Fig. \ref{fig:fig1}a) show that deep velocities in the solar convection zone
are much lower than we would anticipate from our knowledge of simulations.  While the
more recent observations of \citealt{greer&all2015} (Fig. \ref{fig:fig1}b) 
argue that the problem is perhaps not so bad
as it appeared a few years before, observations do not line up with simulations.  The standard
picture of convection from simulations is that deeper motions are driven more intensely, and
thus, through mass conservation, should imprint as the strongest motions at the solar surface.
This is what is seen in simulations, and which was verified by the work of \citealt{lord&all2014}.  However,
this is precisely not what we see at the solar surface.  The horizontal velocity power spectrum
at the solar surface shows excess of power in granular and supergranular scales, but the
``giant cells'' which are theorized to be driven at the base of the 14-density-scale-height-deep
convection zone are not observed \citep{hathaway&all2015}.

This paints a troubling picture.  Years of simulation results -- and the Mixing Length
Theory of convection that they inform -- seem to be wrong.  One fundamental problem is that
we, as a community, are so focused on ensuring that the physics in our simulations are 
as ``correct'' as we can make them that we ignore the fact that we do not understand the
effects of each piece of physics on the nature of convection.  When running large, 3D,
complex simulations, computational cost often limits the number of simulations which can be
run, and parameter space studies are often out of the question.  

Drawing on the knowledge and expertise of those in the physics community who study incompressible,
unstratified \RB convection, we have recently examined hydrodynamic, compressible convection
in simple stratified domains \citealt{anders&brown2017}.  We discovered that these somewhat complicated systems
\emph{transport heat in the same way as incompressible systems}, and that the Mach number of
convection can be easily specified \emph{a priori}.  By setting up a simple experiment, we
were able to learn something concrete about the fundamental nature of convection.  

Here we propose a trio of similar, small experiments.  In each experiment, we test the effects
of one new piece of physics in order to gain an understanding of how they influence the convective
motions and in order to gain a frame of reference for how to understand these effects in 
simulations with more complicated, ``more realistic'' physics.


\begin{figure}[t!]
\centering
%\includegraphics[width=14cm]{figs/Pizzolato_and_D5.eps}
\caption{(a) The big old problem that hanasoge showed, (b) horizontal power spectrum of the
solar surface, (c) something? Superadiabaticity profile?
        \label{fig:fig1}}
\end{figure}



\section{Project 1: Internally heated convection}
Our first experiment involves the study of atmospheres where a constant internal heating
term is included in the energy equation.  This is the simplest form of internal heating,
and is similar to that which has been studied in \RB convection, just with compressibility
and stratification included (e.g., \citealt{goluskin&spiegel2012}).  
We have found that these systems can be constructed so as to
naturally include stable layers underlying unstable, convection regions.  These systems
are important to study, because the changing opacities in the interior of the solar convective
zone effectively act like an internal heating term, depositing energy and driving convection.

We find that these systems essentially have the same input control parameters as simple polytropic
convection (cite), but include a new parameter which determines the depth of the radiative zone
below the convecting region.  We will do runs from low to high Ra for various values of the
depth of the radiative zone and the magnitude of the internal heating to determine the evolved
stratification of these systems and the magnitude of velocity power at the surface of these
simulations.  We anticipate that we will see stuff like in Kapyla 2017.

\section{Project 2: Nonlinear EOS convection}
Most studies of convection are driven either by the choice of fixed boundary conditions
(hot at bottom, cold at top) or through the volumetric depoisition of energy throughout
the domain (internal heating).  Here we propose a project in which we utilize a nonlinear
equation of state to represent the ionization / reionization of hydrogen near the surface of
the Sun.  We examine the effects that this has on the driving boundary layer region near the
top of the simulation, and we do blahblahblah.

\begin{figure}[t!]
\centering
%\includegraphics[width=14cm]{figs/2014_oct_sunspots.jpg}
\caption{(a) word (b) word (c) word
	\label{fig:fig2}}
\end{figure}

\section{Project 3: Realistic Opacities}
While it is convenient to use a constant conductivity (and thus opacity) in the energy
equation, as this makes the interpretation of results very simple, this greatly 
over-simplifies the form of radiative conductivity in natural systems like the Sun.
In this project, we will carefully implement a kramer's opacity -- both fully nonlinear
and initial condition based -- and we will do blahblahblah with it.

\section{Timeline of proposed work}
\paragraph{Year 1:}
\begin{itemize}
\item (Fall '18) Finalize Internally Heated convection work, submit a paper to ApJL.
\item (Spring-Summer '19) Run simulations of nonlinear EOS, write paper, submit to ApJ.
Begin opacity project.
\end{itemize}
%\vspace{-0.5cm}

\paragraph{Year 2:}
\begin{itemize}
\item (Fall '19 - Spring '20) Finalize opacity project, submit ApJ.  Write and
finalize thesis by end of Spring '20.
\end{itemize}

\newpage

\section{Relevance to NASA} 
The proposed work fits with NASA's 2014 Strategic Plan objective
1.4:
``Understand the Sun and its interactions with Earth and the solar
system, including space weather.''
This work also fits in with one of the three overarching science goals
of the Heliophysics section of NASA's 2014 Science Plan: 
``Develop the
knowledge and capability to detect and predict extreme conditions in space to
protect life and society and to safeguard human and robotic explorers beyond
earth.''  Furthermore, simulated observables created by this work will be
directly comparable to data retrieved by the currently operational Solar
Dynamics Observatory and the future NASA-ESA Solar Orbiter mission.

\section{Summary}



\bibliographystyle{apj}
\begingroup
\renewcommand{\section}[2]{}%
\begin{footnotesize}
\bibliography{biblio}
\end{footnotesize}
\endgroup
\end{document}
