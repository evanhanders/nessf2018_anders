\documentclass[aasms,12pt]{article}
\usepackage{natbib}
\setlength{\bibsep}{0pt plus 0.3ex}
\usepackage[margin=0.7in]{geometry}
\usepackage{sectsty}
\usepackage{graphicx}
\usepackage{hyperref}
\usepackage{epstopdf}
\usepackage[skip=2pt,font=small]{caption}
\captionsetup{width=\textwidth}
\usepackage{amssymb, amsmath, amsfonts, xcolor}
\hypersetup{
    colorlinks,
    linkcolor={red!50!black},
    citecolor={blue!80!black},
    urlcolor={blue!80!black}
}


%\usepackage{titlesec}
%
%\titlespacing\section{0pt}{12pt plus 4pt minus 2pt}{5pt plus 2pt minus 2pt}
%\titlespacing\subsection{0pt}{12pt plus 4pt minus 2pt}{0pt plus 2pt minus 2pt}
%\titlespacing\subsubsection{0pt}{12pt plus 4pt minus 2pt}{0pt plus 2pt minus 2pt}


\sectionfont{\normalsize}
%\usepackage{fullpage}


%\citestyle{aa}
\newcommand{\sol}{\ensuremath{\odot}}
\newcommand{\RB}{Rayleigh-B\'{e}nard }
\newcommand{\grad}{\ensuremath{\nabla}}

\usepackage{fancyhdr}
\pagestyle{fancy}
\fancyhf{} % sets both header and footer to nothing
\renewcommand{\headrulewidth}{0pt}





\begin{document}
\begin{center}
   \large\textbf{Towards a more complete understanding of Stratified, Compressible Convection}\\
   \vspace{0.4cm}
   \large{Evan H. Anders}\\
   \vspace{0.4cm}
   \normalsize\textit{Advisor: Benjamin P. Brown}\\
   \normalsize\textit{Laboratory for Atmospheric and Space Physics (LASP) \& University of Colorado at Boulder}\\
\end{center}

\section{Background \& Motivation}
The Sun exhibits active magnetism which cycles in magnitude every 11 years.
This magnetism arises from an 
organized dynamo seated in the turbulent plasma
motions of the solar convective zone, which occupies roughly the outer 30\%
of the Sun's radius. This activity manifests itself in the collection of phenomena generally
referred to as solar activity, including magnetic storms and coronal mass
ejections.  Such activity propagates towards Earth, threatening disruption of 
power grids and aircraft operations as well as endangering astronauts and satellites.
The motivation to understand the Sun's magnetism in our increasingly technological society
is great.  A critical step in protecting our society from the Sun's magnetism is 
to gain an understanding of the nature of the dynamo that generates the Sun's magnetic fields
\cite{nordlund&all2009, charbonneau2014}.

To understand the dynamo that drives the Sun's magnetism, we must understand the convection
which powers that dynamo.  Numerical studies of convection in stratified domain have a
rich history in the past decade.  The early work of \cite{graham1975} and \cite{hurlburt&all1984} and others in simple
domains provided rich insight into the nature of stratified convection and provided a basis
in a field which now regularly creates both complex, 3D global models of convection
(e.g., \cite{brown&all2010}, \cite{guerrero&all2016}, and many others)
and smaller scale local area models with more complex physics 
(e.g., \cite{stein&nordlund2012}, \cite{rempel2014}).
From these efforts we have learned a great deal about the nature of convection, and have
even created beautiful simulations which even \emph{look} like the convection we see on the
surface of the Sun.

Unfortunately, the great advances made in computational prowess within the solar physics
community seem to have surpassed our fundamental knowledge in the field. Recent observations
by Hanasoge et al. (2012) \cite{hanasoge&all2012} (Fig. \ref{fig:fig1}a) show that 
velocities in the solar convection zone
are much lower than we would anticipate from our knowledge of simulations.  While the
more recent observations of Greer et al. (2015) \cite{greer&all2015} (Fig. \ref{fig:fig1}b) 
argue that the problem is perhaps not so bad
as it appeared a few years before.  However, Greer's observations still show a decrease in power at
large length scales, which does not line up with simulations.  Lord et al. (2014) \cite{lord&all2014} showed
that unstable, convecting layers drive motions on a length scale which is proportional to
the local density scale height, which increases with depth.  According to our standard picture of convection
in which the entire convective zone is unstable, the motions driven at high density deep in the convective
zone should be large scale, and should imprint strongly at the solar surface.
However, even doppler measurements of the velocity fields at the solar surface do not show this trend
\cite{hathaway&all2015}
(Fig. \ref{fig:fig1}c).  The motions of surface granules and the slightly deeper supergranules are
clearly present, but the
``giant cells'' which are theorized to be driven at the base of the 14-density-scale-height-deep
convection zone are not observed.

This paints a troubling picture.  Years of simulation results -- and the Mixing Length
Theory (MLT) of convection that they inform -- seem to be wrong.  Recent work by
Brandenburg (2016) \cite{brandenburg2016} has sought to incorporate nonlocal effects in
convection into MLT, but recent simulations \cite{kapyla&all2017} have shown that regions which
carry a great amount of flux convectively do not necessary locally drive the convection.
It is the tendency of simulators in the field to ensuring that the physics in our simulations are 
correct, but this often comes at the price of having testable hypotheses which can be meaningfully
explored. When running complex, 3D simulations, 
computational cost often limits the number of simulations which can be
run, and parameter space studies are often out of the question.  

Drawing on the knowledge and expertise of those in the physics community who study incompressible,
unstratified \RB convection, we have recently examined hydrodynamic, compressible convection
in simple stratified domains \cite{anders&brown2017}.  We discovered that these somewhat complicated systems
\emph{transport heat in the same way as incompressible systems}, and that the Mach number of
convection can be easily specified \emph{a priori}.  By setting up a simple experiment, we
were able to learn something concrete about the fundamental nature of convection.  

Here we propose a trio of similar, small experiments.  In each experiment, we test the effects
of one new piece of physics in order to gain an understanding of how they influence the convective
motions and in order to gain a frame of reference for how to understand these effects in 
simulations with more complicated, ``more realistic'' physics.


\begin{figure}[t]
\centering
\includegraphics[width=\textwidth]{figs/fig1.png}
\caption{(a) Power spectra of solar convective velocities are shown for both observations and simulations,
and both near the surface and deeper \cite{hanasoge&all2012}.  Observations are obtained
using time-distance helioseismology, and show velocities roughly two orders of magnitude lower
than those predicted by simulations, and a decrease in power approaching larger length
scales rather than the opposite. (b) Further observations of solar velocity power  using
ring-diagram helioseismology.  Here, velocity magnitudes are roughly in line with those
predicted by simulations, but show decreasing power as larger scales are approached, unlike
what is expected from simulations \cite{greer&all2015}.  A simple spectrum of horizontal
velocities at the solar surface, obtained using line-of-sight Doppler velocities \cite{hathaway&all2015}.
The length scales of surface granules and deeper supergranules appear as distinct features, but
the hypothesized giant cells are not observed at low wavenumber.
        \label{fig:fig1}}
\end{figure}

\section{Proposed Project}
I propose a series of small projects which will probe and quantify the effects of key
elements of solar convection on the dynamics of the evolved flows.  Each of these pieces
of physics is often included in the more complex global simulations or local area models.

In the interior of the Sun, there is a positive radial gradient of the opacity.  This means
that as one looks further outwards from the center of the Sun, radiation is less efficient at
carrying flux and so convection becomes increasingly essential to carry the solar luminosity.
This convection is strongly driven near the solar photosphere, helped in large part thanks to
the ionization and recombination of hydrogen atoms. We will study both of these phenomena in
small pieces.  

The divergence of the radiative flux in the solar interior acts like an
internal heating term.  Flux is deposited as it ceases to be carried by radiation, and this
deposition of flux either drives convection or requires convection driven nonlocally to carry
it.  We will thus study simple atmospheres which are driven by internal heating rather than
driven by boundary conditions.

The driving of convection at the solar surface by hydrogen ionization and recombination can
be added to a simple study of convection through the use of a phase change at a specific temperature.
We will study atmospheres where one of these phase changes occurs within the domain, and
examine its effects on the dynamics of convection.

Then, we will examine the effects of using a realistic opacity profile, rather than a constant
opacity throughout the depth of the box.  With our knowledge of internally heated convection,
we will understand how the divergence of the radiative flux drives the convection.  With our
knowledge of the hydrogen ionization and recombination, we will be able to employ a mechanism in the
upper atmosphere which efficiently releases the large amount of energy carried by the convection.
By combining these experiments with a more realistic profile of opacity, we will then

The first two projects involve understanding mechanisms of driving convection that are not
frequently used in simple simulations.  The third project involves a careful study of a
more complicated piece of physics which is often included, but whose effects have not
been studied in careful detail.

\subsection{Mini Project 1: Internally heated convection}
Our first experiment involves the study of atmospheres where a constant internal heating
term is included in the energy equation.  This is the simplest form of internal heating,
and is similar to that which has been studied in \RB convection, just with compressibility
and stratification included (e.g., \cite{goluskin&spiegel2012}).  
We have found that these systems can be constructed so as to
naturally include stable layers underlying unstable, convection regions.  These systems
are important to study, because the changing opacities in the interior of the solar convective
zone effectively act like an internal heating term, depositing energy and driving convection.

We find that these systems essentially have the same input control parameters as simple polytropic
convection \cite{anders&brown2017}, which include the Rayleigh number (Ra, the strength of
convective driving and amount of turbulence), the Prandtl number (Pr, the ratio of the thermal
to viscous diffusivity), the characteristic superadiabaticity of the atmosphere (which sets the Mach
number of the flows), and the depth of the atmosphere.  In the internally heated systems, we
find that the magnitude of the internal heating directly relates to the superadiabaticity of the
atmosphere, and thus Ma, but we must split up the depth into two parameters: the depth of the
\emph{unstable} region, and the depth of the new \emph{stable} region.  In \cite{anders&brown2017},
we explored the effects of the magnitude of the superadiabaticity on the resulting convection,
and found that -- except where the Mach number is very high -- it has almost no effect on the
other properties of the convective flows.  

In this project, we will focus on understanding how the depth of the atmosphere, and the amount
of stratification felt by the flows, affects the resultant convection.  We will set the depth
of the convection zone (to 1, 3, and then 5 density scale heights), and for each of these depths
we will run simulations for shallow and deep radiative zones.  We will examine the evolved
stratification of these atmospheres in order to determine where convection is driven and where
it is not.  Preliminary results show that the so-called ``Deardorff zones'' present in the
simulations of \cite{kapyla&all2017} are also present in these simple simulations of
internally heated convection and do not require complex forms of the conductivity to be achieved.
We expect that the extent of these subadiabatic zones in which convection carries the flux can
be determined \emph{a priori} in these simple simulations, but further work must be done.

We're looking for the motions at the surface and also the stratification.


Our research group has extensive experience studying convection penetrating into stable layers.
In soon-to-be-submitted work (Brown et al. 2018, in prep) a systematic approach to studying
penetrative convection is laid out, and it is found that the depth of penetration is well
predicted by a buoyancy equilibration model.  We anticipate that the same will be true in 
Internaly heated convection.  Another piece of forthcoming work (Oishi et al. 2018, in prep)
shows that penetrating convection does not extend the convection zone downwards into the stable
layer, as typical knowledge assumes.  It rather dredges up excess entropy into the convecting region
and makes the bottom of the convecting region marginally stable.  A similar effect was found
recently in more complex systems \cite{kapyla&all2017}, and is also observed in our preliminary
runs of internally heated convection.  This work will study this phenomenon carefully and apply
this knowledge to the Solar Convective Conundrum.

\subsection{Mini Project 2: Hydrogen recombination}
Most studies of convection are driven either by the choice of fixed boundary conditions
(hot at bottom, cold at top) or through the volumetric depoisition of energy throughout
the domain (internal heating).  Here we propose a project in which we utilize a nonlinear
equation of state to represent the ionization / reionization of hydrogen near the surface of
the Sun.  We examine the effects that this has on the driving boundary layer region near the
top of the simulation, and we do blahblahblah.

In order to carefully study the effects of the ionization of hydrogen, or other similar phase
changes, we must first understand the reference state for such an atmosphere.  It will no longer
simply be polytropic, as the nonlinear EOS rules out the simple polytropic assumption. Once the
appropriate state which is in hydrostatic and thermal equilibrium despite this phase change is discovered,
we will then carefully study the effects of the temperature of ionization (which will in turn
determine its depth), and also the ionization energy of the transition.  In standard simulations
of convection, the interactions of the flows with a hard boundary forms a thin boundary layer,
which scales downwards as the diffusivities shrink, and this boundary generally drives convection.
We suspect that we can create atmospheres with larger or smaller boundary layers than the natural
thermal boundary length scale, which will nominally drive convection on different length scales
than the natural one.  We are interested in seeing how these ``too large'' convective flows
interact with their surroundings, and in seeing how the atmosphere naturally evolves in the presence
of these flows.

We are particularly interested in determining the effects of \emph{where} in the atmosphere the
transition from ionized to neutral Hydrogen occurs, and also \emph{how much energy} is involved in
the ionization process.  These are two simple controls which can be examined in full detail through
a suite of simulations.  We hypothesize that, for sufficiently energetic ionization processes (such as
that of neutral hydrogen), a natural boundary layer will form between an overlying stable layer and
underlying convecting region in the atmosphere.  We are further interested in determining how the
ionization energy determines the length scale of the boundary layer.

Using these two control knobs, we are very interested in determining the startification of the evolved
solution.  This will tell us how large of a region is driving convection (is it a small layer 
near the reionization, or does it extend to a great depth below that?  We are also very interested
in the filling factor of convection compared to simple boundary-driven convection.


%\begin{figure}[t!]
%\centering
%%\includegraphics[width=14cm]{figs/2014_oct_sunspots.jpg}
%\caption{(a) word (b) word (c) word
%	\label{fig:fig2}}
%\end{figure}

\subsection{Mini Project 3: Kramer's opacity}
While it is convenient to use a simple form of the radiative flux of the form of
Fourier's law of conduction \cite{lecoanet&all2014} (flux $\propto -\kappa \grad T$), as such a formulation
makes it very simple to understand what regions in the atmosphere are stable and which
are unstable, and exactly \emph{how} unstable they are, things are not so simple in nature.
While the general form is correct, $\kappa$ is, in general, not a constant value through the
depth of the atmosphere, but rather takes the form of a Kramer's opacity, such as 
$\kappa \propto T^3 / \rho$. Building upon our previous work, we will determine the proper
adiabatic gradient for a system with this form of radiative flux, understand how to quantify
heat transport and determine how to set the Mach number of the experiment.  Then, we will run experiments 
at low and high Mach number.  We will compare a $\kappa$ which is allowed to change with time
and a $\kappa$ that changes with height, but not time.  We hypothesize that, for low Mach number
flows, the time variance of $\kappa$, which makes it a fully nonlinear term, is unimportant, as
low Ma flows are by definition very small compared to the background.  Once we understand which
regimes it is appropriate to use a fluctuating verses non-fluctuating $\kappa$, we will
as further questions.

We are curious to see if this more realistic form of the conductivity significantly changes
the transport properties of the overall convection: does it affect the heat transport
(Nu)?  How does the change in the adiabatic temperature gradient under this formulation
change our intuition?  We can likely naturally have heavily driven regions and lightly
driven regions of naturally-occuring, internally heated convection.  Our studies from project
1 will help inform how to interpret the divergence of this flux term as an internal heating
source.

\section{Numerical Tools and Feasibility}
I will use the open-source Dedalus\footnote{\url{http://dedalus-project.org/}} pseudospectral framework 
\cite{burns&all2016} to carry out my simulations.  
Dedalus is a flexible solver of general partial differential equations,
making it extremely easy to study diverse sets of equations under many different atmospheric
constraints.  I have already published one paper using this tool \cite{anders&brown2017},
have submitted another paper, and am now adept at using it to create suites of simulations
in short timeframes.

I will primarily study 2D convective solutions in plane-parallel atmospheres in order to gain
intuition about the mean behavior of vertical profiles within the atmosphere.  Once I have a grasp
on how my measurements vary in 2D across parameter space, I will run select 3D simulations to
verify whether or not that behavior holds in 3D, as I did in my previous paper \cite{anders&brown2017}.
In cases where 2D and 3D diverge, I will quantify how and why they do so, but most questions I am
asking are quite basic, and most of the systems I propose to study here have not been studied
in the compressible context, at least not recently.  By primarily studying in 2D, and by carefully
selecting my 3D runs once I know which parameters I must examine more carefully, I can complete
a full suite of simulations, such as those in my previous paper \cite{anders&brown2017}, using
roughly 3 million CPU-hours.  Through my advisor, I have access to an allocation on NASA Pleiades
of roughly 20 million CPU-hours/year, so one- or two- of the following projects of the scope I am
proposing can be completed each year.

Furthermore, my recent work (Anders, Brown, \& Oishi 2018, submitted to PRFluids) has
shown that properly constructed boundary value problems, coupled with initial value problems,
can fast-forward the slow thermal evolution of these convective simulations.  This work was done in 
Boussinesq, \RB convection but can be easily extended to stratified convection, and will greatly
extend both the number of simulations we are able to complete and the level of turbulent driving
(while attaining converged atmospheres) that we are able to solve.



\subsection{Timeline of proposed work}
\textbf{Year 1 (Fall 2018 - Summer 2019):}
\begin{itemize}
\vspace{-0.2cm}
\item Finalize work on Internally Heated convection, which will be started Spring-Summer 2018.
Submit a short paper summarizing the methods and results of this work to the Astrophysical Journal
by mid-Fall 2018.  Release code upon submission such that the community can use it.
\vspace{-0.2cm}
\item Delve into literature on past work regarding ionizing convection and moist convection.
Understand the pieces of physics necessary to correctly implement a nonlinear equation of state.
Begin to develop Dedalus simulations of ionizing convection by end of 2018.
\vspace{-0.2cm}
\item Fully Develop ionizing convection code.  Determine the range of parameter space to be studied
and execute the simulations within this range.  Analyze data, and have a short paper written and
submitted on them to the Astrophysical Journal by the end of summer 2019.
\end{itemize}

\textbf{Year 2 (Fall 2019 - Spring 2020):}
\begin{itemize}
\vspace{-0.2cm}
\item Begin literature review on Kramer's opacity late summer 2019-early fall 2020.  Understand
past work done, and implementing both time-dependent (fully nonlinear) and time-independent
versions of the Kramer's opacity in simple atmospheres.  Run a suite of simulations at high-
and low- Mach number by end of year 2019.
\vspace{-0.2cm}
\item Analyze data from Kramer's opacity results and prepare a small paper to submit to the
Astrophysical Journal Letters by end of Winter 2020.
\vspace{-0.2cm}
\item Combine work from five published papers into a thesis, to be defended at the end of 
Spring 2020.
\end{itemize}

\section{Relevance to NASA} 
The proposed work fits with NASA's 2014 Strategic Plan objective
1.4:
``Understand the Sun and its interactions with Earth and the solar
system, including space weather.''  Specifically, I aim to help answer
the fundamental question, ``What causes the Sun to vary?'' 
This work also aims to answer one of the three overarching science goals
in chapter 4.1 of NASA's 2014 Science Plan: 
``Develop the
knowledge and capability to detect and predict extreme conditions in space to
protect life and society and to safeguard human and robotic explorers beyond
earth.'' In order to understand how to predict space weather appropriately, we
need to understand the processes that cause this weather.  It is clear from
recent work that our understanding of the fundamentals of convection is not as perfect
as we once thought, and now is an exciting time to clarify our theory and determine which
parts of it fail and which parts hold true under more examination.  Only once we understand
the fundamental nature of stratified, compresible convection can we begin to understand
how it drives the dynamo in the Sun in the presence of many complications such as
differential rotation, shear layers near the base and top of the convection zone, and
magnetism.

The work has been motivated by data from the Helioseismic and Magnetic Imager (HMI) onboard
the NASA Solar Dynamics Observtory (SDO) spacecraft 
\cite{hanasoge&all2012, greer&all2015, hathaway&all2015}, and will continue to be informed by
new helioiseismic measurements made from SDO data, and from the new measurements which will
be made possible by the upcoming joint NASA-ESA Solar Orbiter's Polarimetric and 
Helioseismic Imager (PHI).


\section{Summary}
Recent observations call into question our fundamental understanding of stratified
convection in systems such as the solar convection zone \cite{hanasoge&all2012, greer&all2015}.
Here we present three focused, scoped studies of stratified convection which probe the
specific effects of individual elements of convection in the Sun.  Convection must carry
flux in the Sun's convective envelope, because the radiative flux becomes too small and deposits
energy there, acting essentially like internal heating. Convection is driven at the surface of
the Sun due to the ionization and recombination of hydrogen near the solar photosphere.  The
magnitude of the flux that must be carried by convection varies greatly throughout the depth
of the convection zone due to the increase of opacity with height.

We propose to probe the first of these elements by studying simple internal heating systems,
the second of these elements by studying convection with a nonlinear equation of state / phase
change, and the third of these elements by studying convection with a realistic opacity profile
in the context of our knowledge from the first two experiments.  Due to the developed nature of
our computational tool, Dedalus, the simulations for these projects can be implemented and
carried out on short timescales, and the body of work suggested here should be finished within
two years, by the end of the spring of 2019.


\bibliographystyle{apj}

\bibliography{biblio}
\end{document}
