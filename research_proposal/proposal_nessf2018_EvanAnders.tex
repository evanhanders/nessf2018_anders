\documentclass[aasms,12pt]{article}
\usepackage{natbib}
\setlength{\bibsep}{0pt plus 0.3ex}
\usepackage[margin=1in]{geometry}
\usepackage{sectsty}
\usepackage{graphicx}
\usepackage{epstopdf}
\usepackage[skip=2pt,font=small]{caption}
\captionsetup{width=\textwidth}


%\usepackage{titlesec}
%
%\titlespacing\section{0pt}{12pt plus 4pt minus 2pt}{5pt plus 2pt minus 2pt}
%\titlespacing\subsection{0pt}{12pt plus 4pt minus 2pt}{0pt plus 2pt minus 2pt}
%\titlespacing\subsubsection{0pt}{12pt plus 4pt minus 2pt}{0pt plus 2pt minus 2pt}


\sectionfont{\normalsize}
%\usepackage{fullpage}


%\citestyle{aa}
\newcommand{\sol}{\ensuremath{\odot}}
\newcommand{\RB}{Rayleigh-B\'{e}nard }


\usepackage{fancyhdr}
\pagestyle{fancy}
\fancyhf{} % sets both header and footer to nothing
\renewcommand{\headrulewidth}{0pt}
\cfoot{\thepage}
\rfoot{\footnotesize{Evan H. Anders, NASA NESSF 2018}}




\begin{document}
\begin{center}
   \large\textbf{Towards a more complete understanding of Stratified, Compressible Convection}\\
   \vspace{0.4cm}
   \large{Evan H. Anders}\\
   \vspace{0.4cm}
   \normalsize\textit{Advisor: Benjamin Brown}\\
   \normalsize\textit{LASP, University of Colorado at Boulder}\\
\end{center}



\abstract{Blah blah blah. This project supports objective 1.4 of NASA's 2014 Strategic Plan
	and will assist in developing ``the
	knowledge and capability to detect and predict extreme conditions in
	space'' in accordance with NASA's overarching Heliophysics science goals.}



\section{Introduction}
The Sun exhibits active magnetism which cycles in magnitude every 11 years.
This magnetism arises from an 
organized dynamo seated in the turbulent plasma
motions of the solar convective zone, which occupies roughly the outer 30\%
of the Sun's radius. This activity manifests itself in the collection of phenomena generally
referred to as solar activity, including magnetic storms and coronal mass
ejections.  Such activity propagates towards Earth, threatening disruption of 
power grids and aircraft operations as well as endangering astronauts and satellites.
The motivation to understand the Sun's magnetism in our increasingly technological society
is great.  A critical step in protecting our society from the Sun's magnetism is 
to gain an understanding of the nature of the dynamo that generates the Sun's magnetic fields
\citep{nordlund&all2009, charbonneau2014}.

To understand the dynamo that drives the Sun's magnetism, we must understand the convection
which powers that dynamo.  Numerical studies of convection in stratified domain have a
rich history in the past decade.  The early work of \citealt{graham1975} and \citealt{hurlburt&all1984} and others in simple
domains provided rich insight into the nature of stratified convection and provided a basis
in a field which now regularly creates both complex, 3D global models of convection
(e.g., \citealt{brown&all2010}, \citealt{guerrero&all2016}, and many others)
and smaller scale local area models with more complex physics 
(e.g., \citealt{stein&nordlund2012}, \citealt{rempel2014}).
From these efforts we have learned a great deal about the nature of convection, and have
even created beautiful simulations which even \emph{look} like the convection we see on the
surface of the Sun.

Unfortunately, the great advances made in computational prowess within the solar physics
community seem to have surpassed our fundamental knowledge in the field. Recent observations
by \citealt{hanasoge&all2012} (Fig. \ref{fig:fig1}a) show that deep velocities in the solar convection zone
are much lower than we would anticipate from our knowledge of simulations.  While the
more recent observations of \citealt{greer&all2015} (Fig. \ref{fig:fig1}b) 
argue that the problem is perhaps not so bad
as it appeared a few years before, observations do not line up with simulations.  The standard
picture of convection from simulations is that deeper motions are driven more intensely, and
thus, through mass conservation, should imprint as the strongest motions at the solar surface.
This is what is seen in simulations, and which was verified by the work of \citealt{lord&all2014}.  However,
this is precisely not what we see at the solar surface.  The horizontal velocity power spectrum
at the solar surface shows excess of power in granular and supergranular scales, but the
``giant cells'' which are theorized to be driven at the base of the 14-density-scale-height-deep
convection zone are not observed \citep{hathaway&all2015}.

This paints a troubling picture.  Years of simulation results -- and the Mixing Length
Theory of convection that they inform -- seem to be wrong.  One fundamental problem is that
we, as a community, are so focused on ensuring that the physics in our simulations are 
as ``correct'' as we can make them that we ignore the fact that we do not understand the
effects of each piece of physics on the nature of convection.  When running large, 3D,
complex simulations, computational cost often limits the number of simulations which can be
run, and parameter space studies are often out of the question.  

Drawing on the knowledge and expertise of those in the physics community who study incompressible,
unstratified \RB convection, we have recently examined hydrodynamic, compressible convection
in simple stratified domains \citealt{anders&brown2017}.  We discovered that these somewhat complicated systems
\emph{transport heat in the same way as incompressible systems}, and that the Mach number of
convection can be easily specified \emph{a priori}.  By setting up a simple experiment, we
were able to learn something concrete about the fundamental nature of convection.  

Here we propose a trio of similar, small experiments.  In each experiment, we test the effects
of one new piece of physics in order to gain an understanding of how they influence the convective
motions and in order to gain a frame of reference for how to understand these effects in 
simulations with more complicated, ``more realistic'' physics.


\begin{figure}[t!]
\centering
%\includegraphics[width=14cm]{figs/Pizzolato_and_D5.eps}
\caption{(a) The big old problem that hanasoge showed, (b) horizontal power spectrum of the
solar surface, (c) something? Superadiabaticity profile?
        \label{fig:fig1}}
\end{figure}

\section{Numerical Tools}
I will use Dedalus, because it's awesome and it allows us to study many flexible atmospheres
under many equation sets easily.

I'll primarily use 2D simulations to get intuition about the answers I'm trying to solve,
then verify them in 3D, as I did in my paper.  The sims shown there took about 3M CPU hours
to complete, and I have access to an allocation of ~20M CPU hours / year on Pleiades, which
means that I can complete a few projects of that scope within a year's time.

Furthermore, recent work (Anders, Brown, \& Oishi 2018, submitted to PRFluids) has been
done to use boundary value problems along with initial value problems to fast-forward
the slow thermal evolution of these convective simulations.  This work was done in 
Boussinesq convection but can be easily extended to stratified convection, and will greatly
extend the number of simulations we will be capable of performing given our modest allocation.


\section{Project 1: Internally heated convection}
Our first experiment involves the study of atmospheres where a constant internal heating
term is included in the energy equation.  This is the simplest form of internal heating,
and is similar to that which has been studied in \RB convection, just with compressibility
and stratification included (e.g., \citealt{goluskin&spiegel2012}).  
We have found that these systems can be constructed so as to
naturally include stable layers underlying unstable, convection regions.  These systems
are important to study, because the changing opacities in the interior of the solar convective
zone effectively act like an internal heating term, depositing energy and driving convection.

We find that these systems essentially have the same input control parameters as simple polytropic
convection \citep{anders&brown2017}, which include the Rayleigh number (Ra, the strength of
convective driving and amount of turbulence), the Prandtl number (Pr, the ratio of the thermal
to viscous diffusivity), the characteristic superadiabaticity of the atmosphere (which sets the Mach
number of the flows), and the depth of the atmosphere.  In the internally heated systems, we
find that the magnitude of the internal heating directly relates to the superadiabaticity of the
atmosphere, and thus Ma, but we must split up the depth into two parameters: the depth of the
\emph{unstable} region, and the depth of the new \emph{stable} region.  In \citealt{anders&brown2017},
we explored the effects of the magnitude of the superadiabaticity on the resulting convection,
and found that -- except where the Mach number is very high -- it has almost no effect on the
other properties of the convective flows.  

In this project, we will focus on understanding how the depth of the atmosphere, and the amount
of stratification felt by the flows, affects the resultant convection.  We will set the depth
of the convection zone (to 1, 3, and then 5 density scale heights), and for each of these depths
we will run simulations for shallow and deep radiative zones.  We will examine the evolved
stratification of these atmospheres in order to determine where convection is driven and where
it is not.  Preliminary results show that the so-called ``Deardorff zones'' present in the
simulations of \citealt{kapyla&all2017} are also present in these simple simulations of
internally heated convection and do not require complex forms of the conductivity to be achieved.
We expect that the extent of these subadiabatic zones in which convection carries the flux can
be determined \emph{a priori} in these simple simulations, but further work must be done.

We're looking for the motions at the surface and also the stratification.

\section{Project 2: Nonlinear EOS convection}
Most studies of convection are driven either by the choice of fixed boundary conditions
(hot at bottom, cold at top) or through the volumetric depoisition of energy throughout
the domain (internal heating).  Here we propose a project in which we utilize a nonlinear
equation of state to represent the ionization / reionization of hydrogen near the surface of
the Sun.  We examine the effects that this has on the driving boundary layer region near the
top of the simulation, and we do blahblahblah.

In order to carefully study the effects of the ionization of hydrogen, or other similar phase
changes, we must first understand the reference state for such an atmosphere.  It will no longer
simply be polytropic, as the nonlinear EOS rules out the simple polytropic assumption. Once the
appropriate state which is in hydrostatic and thermal equilibrium despite this phase change is discovered,
we will then carefully study the effects of the temperature of ionization (which will in turn
determine its depth), and also the ionization energy of the transition.  In standard simulations
of convection, the interactions of the flows with a hard boundary forms a thin boundary layer,
which scales downwards as the diffusivities shrink, and this boundary generally drives convection.
We suspect that we can create atmospheres with larger or smaller boundary layers than the natural
thermal boundary length scale, which will nominally drive convection on different length scales
than the natural one.  We are interested in seeing how these ``too large'' convective flows
interact with their surroundings, and in seeing how the atmosphere naturally evolves in the presence
of these flows.

Things we can study: stratification of evolved solution.  Stable layers?  filling factors.
Power spectrum.

\begin{figure}[t!]
\centering
%\includegraphics[width=14cm]{figs/2014_oct_sunspots.jpg}
\caption{(a) word (b) word (c) word
	\label{fig:fig2}}
\end{figure}

\section{Project 3: Realistic Opacities}
While it is convenient to use a constant conductivity (and thus opacity) in the energy
equation, as this makes the interpretation of results very simple, this greatly 
over-simplifies the form of radiative conductivity in natural systems like the Sun.
In this project, we will carefully implement a kramer's opacity of the form
$\kappa \propto T^3 / \rho$.  We will run experiments at low and high Mach number,
some in which $\kappa$ is constant with time and some in which $\kappa$ is allowed to
evolved with time.  We anticipate that at low Mach number, the deviations from the initial state
will be small, and the fully nonlinear effects of this term will not be important.  At high
Mach number, however, like the solar surface, we expect this term to vary significantly
from the high temperature, low density upflows to the low temperature, high density downflows.

We are curious to see if this more realistic form of the conductivity significantly changes
the transport properties of the overall convection: does it affect the heat transport
(Nu)?  How does the change in the adiabatic temperature gradient under this formulation
change our intuition?  We can likely naturally have heavily driven regions and lightly
driven regions of naturally-occuring, internally heated convection.  Our studies from project
1 will help inform how to interpret the divergence of this flux term as an internal heating
source.

\section{Timeline of proposed work}
\paragraph{Year 1:}
\begin{itemize}
\item (Fall '18) Finalize Internally Heated convection work, submit a paper to ApJL.
\item (Spring-Summer '19) Run simulations of nonlinear EOS, write paper, submit to ApJ.
Begin opacity project.
\end{itemize}
%\vspace{-0.5cm}

\paragraph{Year 2:}
\begin{itemize}
\item (Fall '19 - Spring '20) Finalize opacity project, submit ApJ.  Write and
finalize thesis by end of Spring '20.
\end{itemize}

\newpage

\section{Relevance to NASA} 
The proposed work fits with NASA's 2014 Strategic Plan objective
1.4:
``Understand the Sun and its interactions with Earth and the solar
system, including space weather.''
This work also fits in with one of the three overarching science goals
of the Heliophysics section of NASA's 2014 Science Plan: 
``Develop the
knowledge and capability to detect and predict extreme conditions in space to
protect life and society and to safeguard human and robotic explorers beyond
earth.''  Furthermore, simulated observables created by this work will be
directly comparable to data retrieved by the currently operational Solar
Dynamics Observatory and the future NASA-ESA Solar Orbiter mission.

\section{Summary}



\bibliographystyle{apj}
\begingroup
\renewcommand{\section}[2]{}%
\begin{footnotesize}
\bibliography{biblio}
\end{footnotesize}
\endgroup
\end{document}
