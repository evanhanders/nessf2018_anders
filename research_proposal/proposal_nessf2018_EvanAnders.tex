\documentclass[aasms,12pt]{article}
\usepackage{natbib}
\setlength{\bibsep}{0pt plus 0.3ex}
\usepackage[margin=1in]{geometry}
\usepackage{sectsty}
\usepackage{graphicx}
\usepackage{hyperref}
\usepackage{epstopdf}
\usepackage[skip=2pt,font=small]{caption}
\captionsetup{width=\textwidth}
\usepackage{amssymb, amsmath, amsfonts, xcolor}
\hypersetup{
    colorlinks,
    linkcolor={red!50!black},
    citecolor={blue!80!black},
    urlcolor={blue!80!black}
}


\sectionfont{\normalsize}
\subsectionfont{\small}


%\citestyle{aa}
\newcommand{\sol}{\ensuremath{\odot}}
\newcommand{\RB}{Rayleigh-B\'{e}nard }
\newcommand{\grad}{\ensuremath{\nabla}}

\usepackage{fancyhdr}
\pagestyle{fancy}
\fancyhf{} % sets both header and footer to nothing
\renewcommand{\headrulewidth}{0pt}





\begin{document}
\begin{center}
   \large\textbf{Towards a more complete understanding of solar convection}\\
   \vspace{0.4cm}
   \large{Evan H. Anders}\\
   \vspace{0.4cm}
   \normalsize\textit{Advisor: Benjamin P. Brown}\\
   \normalsize\textit{Laboratory for Atmospheric and Space Physics (LASP) \& University of Colorado at Boulder}\\
\end{center}

\section{Background \& Motivation}
The Sun exhibits a 22-year active magnetic cycle.
An organized dynamo seated in the turbulent plasma
motions of the solar convective zone drives this magnetism.
Solar magnetism manifests itself in the collection of phenomena generally
referred to as solar activity, including magnetic storms and coronal mass
ejections.  Energetic particles ejected by this activity propagate towards Earth, 
threatening to disrupt power grids and aircraft operations in addition to endangering astronauts and satellites.
Understanding the nature of the dynamo that generates the Sun's magnetic fields is a critical
step to protecting our society from the threats of solar activity \citep{charbonneau2014}.

The solar dynamo is powered by convection, and
an understanding of that convection is essential to understanding
how solar magnetism is generated. The early work of \cite{graham1975}, \cite{hurlburt&all1984},
and others who studied convection in plane-parallel atmospheres
provided rich insight into the nature of solar-like stratified convection.  From this basis,
the field has blossomed into one which now regularly creates complex, 3D global models of convectively-driven
dynamos (e.g., \cite{brown&all2010} and \cite{guerrero&all2016})
and smaller scale local area models with more complex physics 
(e.g., \cite{stein&nordlund2012} and \cite{rempel2014}).
These efforts have taught us a great deal about the nature of convection, and
beautiful modern simulations even visually resemble the convection observed on the solar surface.

Unfortunately, the great advances made in computational prowess within the solar convection
community have surpassed our fundamental knowledge in the field. This is clear in the
``Solar Convective Conundrum,'' in which observations and theory starkly disagree.
The helioseismic measurements of \cite{hanasoge&all2012} and \cite{greer&all2015} (Fig. \ref{fig:fig1}a)
showed an absence of power in the solar velocity spectrum at large length scales.  Simulations
hypothesized that large-scale
``giant cells'' should be driven by deep convective motions and visible throughout the solar convective
zone, but we do not see these giant cells.
Even simpler doppler measurements of the velocity fields at the solar surface, which are not muddied
by complex helioseismic inversions, lack
giant cells (\citealt{hathaway&all2015} \& Fig. \ref{fig:fig1}b).  
The motions of surface granules and the slightly deeper supergranules are clearly present, but no
larger length scale is distinct.  These combined observational inferences make clear that the lack of
giant cells remains a conundrum and deserves further exploration.

\begin{figure}[t]
\centering
\includegraphics[width=\textwidth]{figs/fig1.png}
\caption{(a) Ring-diagram helioseismic observations of the solar velocity power spectrum at various depths.  
Here, velocity magnitudes are roughly in line with those
predicted by simulations, but show decreasing power as larger scales are approached, unlike
what is expected from simulations \cite{greer&all2015}.  (b) A simple spectrum of horizontal
velocities at the solar surface, obtained using line-of-sight Doppler velocities \cite{hathaway&all2015}.
The length scales of surface granules and deeper supergranules appear as distinct features, but
the hypothesized giant cells are not observed at low wavenumber.
        \label{fig:fig1}}
\end{figure}

The lack of giant cells is disturbing, as is exemplified by the work of \cite{lord&all2014}.  They
performed radiative magnetohydrodynamic simulations of the solar photosphere and convective zone using the
MURaM code.  These simulations
showed that the length scale of convective motions is determined by the depth in the atmosphere at 
which they are driven. Deep convection drives larger scale motions due to the increasing nature of the
local density scale height with depth. These deep motions which are driven at high density should imprint
hugely on the surface motions.
The lack of their presence in our observations implies that our models are missing a piece of physics. 
Either processes in the solar convection zone mask the deeper motions, or \emph{they are never driven
in the first place}.

In order to determine the fate of giant cells, there is a need to return to simple convective models to
understand how the presence of convection modifies the underlying atmospheric stratification, and how
this stratification in turn affects the observable convection at the surface of the atmosphere.
Here I present two simple, small experiments which aim to elucidate fundamental aspects of solar
convection, and to determine whether either of these mechanisms is the reason that we do not see giant cells
at the solar surface. I will use methods similar to those I employed in my recently published work
\citep{anders&brown2017} to create simple, controlled experiments studying turbulent
compressible convection in stratified atmospheres 
to gain a deep understanding of the underlying convective physics.
Furthermore, the knowledge gained in forthcoming work (Anders, Brown, \& Oishi 2018, submitted to PRFluids),
in which we discovered a mechanism for self-consistently converging convective simulations on short timescales,
will be used to make these projects manageable on human timescales.

\section{Proposed Project}
In simulations of convection, motions are often driven by enforced boundary conditions
on the thermodynamic state.  Boundary layers at the top and bottom of the atmosphere naturally arise,
and convection is strongly driven within those boundary layers.  Convective driving in the Sun is more
complex. A positive radial gradient of opacity within the Sun decreases the efficiency with which
radiation can carry the solar luminosity with increasing height in the convective zone.
This results in a divergence of radiative flux which deposits
energy in the convective layers, and this energy must in turn be carried by convective motions.
In other words, the convection in the Sun is not driven by a sharp lower boundary but rather
through naturally occuring internal heating.
Further, the upper boundary layer of the solar convective zone does not arise because of a hard
upper boundary, but rather because of radiative losses at the photosphere, paired with the ionization and 
recombination of hydrogen. These
two effects -- the internal driving of solar convection and the driving of convection at the solar surface 
by hydrogen ioniziation -- have not been studied carefully in at least two decades.

Recent exciting work by \cite{kapyla&all2017}
exhibited convection zones in which deep layers of the convection
zone are stable, a setup in which giant cells would not be driven.  
The authors attribute these deep, stable layers to their inclusion of Kramers' opacity
effects which drives convection internally in a manner similar to that in the Sun.  
However, forthcoming work by my advisor and collaborators (Brown et al. 2018 in prep, Oishi et al. 2018 in prep) 
shows that 
stable lower-layers of convective zones naturally arise where convective zones overly stable regions
and convective motions are at low Mach number.
Further, first results from work that I am conducting on stratified, internally
heated convection (modeled after simpler studes, e.g., \cite{goluskin&spiegel2012}),
show that these stable convecting layers arise naturally when internal heating is the mechanism which
drives convection, even in the presence of a constant radiative conductivity throughout the depth of the atmosphere.

Since the process of internal heating -- not the complex form of a Kramers' opacity --
appears to be the fundamental cause of the stratification effects seen recently by \cite{kapyla&all2017},
the first project that I propose in year 1 is a careful study of the effects of Kramers' opacity on convection.
The second project I propose in year 2 is a careful study of the effects of hydrogen ionization near the surface
of the Sun, building on the previous work of e.g., \cite{rast&toomre1993}.

\subsection{Project 1: Effect of Kramers' opacity on solar convection}
Many careful studies of convection employ a constant radiative conductivity.
The transport of heat within an optically thick atmosphere, in the absence of convective transport,
is often quantified by Fourier's law of conduction \citep{lecoanet&all2014}, in which the radiative
flux is proportional to the conductivity and the temperature gradient.  
While a constant conductivity in time and space allows for the creation of simple measurements of the
heat transport in the evolved atmosphere compared to the initial atmosphere, such an assumption about
the conductivity is coupled with unrealistic assumptions regarding the functional form of the opacity.  

While a constant radiative
conductivity is the go-to choice for many in the physics community who study incompressible
\RB convection, it is often not the choice for those in the heliophysics or astrophysics
communities.  Instead, these communities generally employ a radiative conductivity which is a function
of the temperature, density, and the Kramers' opacity \citep{barekat&brandenburg2014, brandenburg2016, kapyla&all2017}.
This results in a vastly varying conductivity throughout the depth of the atmosphere, which makes the
interpretation of the solution much more difficult but which more carefully captures the physics of nature.

In order to study the importance of Kramers' opacity, a frame of reference must be constructed
in which to study this varying opacity. For example, what atmospheric parameter determines the Mach number
of evolved flows?  At what value of the Rayleigh number
does convection turn on (and thus, at what \emph{supercriticality} are other studies
being run)?  What are the parameters of the initial state that determine key quantities
of the evolved convection, and what can we learn about the evolved convection from them?
Only by answering these simple questions can a careful study of Kramers' Opacity be carried out.
Through answering these questions about basic polytropic systems, I was able to determine that
regardless of Mach number, basic stratified compressible convection \emph{transports heat
in the same manner as unstratified, incompressible \RB convection} \citep{anders&brown2017}.

After determining how to carry out controlled experiments studying the role of Kramers' opacity,
I will study the importance of the nonlinear nature of this opacity on convection.
In downflows (where density is high and temperature is low), the radiative conductivity
should be low compared to upflows.
However, at low Mach number, where variations in $T$ and $\rho$ are small
in upflows and downflows, we anticipate that this effect will be unimportant.  In the Sun,
the Mach number of convection ranges from nearly Mach 1 at the surface to very low Mach number
deep in the interior, where Mach number is O($10^{-5}$).  Thus, it is important to understand how this
complex form of opacity interacts with convection at both high and low Mach number in order to understand
how it influences solar convection.

In summary, my goal in year 1 is to quantify the importance of nonlinearities in the opacity felt by
solar convection.  My motivation is to understand the importance of
these nonlinearities on nonlocal convective transport and atmospheric stratification.  If
comparatively high opacity in the downflows enhances the importance of nonlocal transport at
all Mach numbers, then the ``entropy rain'' addition to convective theory expanded by
\cite{brandenburg2016} and explored by \cite{kapyla&all2017} could be an essential element
of solar convection, and this nonlocal transport could drastically change the stratification of
deep convection where low-entropy fluid falls and then resides.  If this effect consistently leads
to marginal stability in the deep convection zone, this could be the right explanation for the lack
of observed giant cells.

\subsection{Project 2: Solar convection influenced by Hydrogen ionization and recombination}
Convection is strongly driven at the solar surface by the ionization and recombination of hydrogen.
This piece of physics is absent from many studies of solar convection. Instead, surface convection
is often driven by either an imposed entropy draining layer at the upper boundary \citep{kapyla&all2017}, 
or the natural thermal boundary layer that forms near the upper surface \citep{anders&brown2017}.
These methods have a considerable problem in that the size of low entropy
convective elements which form at the surface are determined either by the pre-imposed size of the
entropy draining region or the natural size of the thermal boundary layer (which depends on
the opacity).

The scale of convective driving near the photosphere of the Sun is \emph{much} larger than the
natural thermal diffusive length scale. The solar surface convective boundary layer, which is
much larger than the local thermal diffusion length scale, is
determined at least in part by the depth at which hydrogen ionizes and recombines.
The large convective boundary near the solar surface drives relatively large convective elements,
and the length scale of those elements likely plays an important role on the convective dynamics. 
If the size of these elements allows them to persist deep in the atmosphere, they could significantly alter the
mean stratification deep in the convective zone.  We aim to study the difference in the 
nature of convection -- especially in the surface power spectrum -- when hydrogen recombination
is the driver of convection near the upper boundary, rather than the other more common methods.

I will implement the basic nature of hydrogen ionization and recombination through the use of
a nonlinear equation of state built around a single-atmoic level model of particles,
similar to that used in \cite{rast&toomre1993}.
This work in year 2 will further be guided by prior studies on moist convection (e.g., \citet{leconte&all2017}),
in which phase changes resulting in cloud formation are studied. We will study the effects of
hydrogen ionization on atmospheric stratification in two ways.  First, we will determine the effects of the
\emph{location} of the ionizing layer in order to determine if we can naturally make transitions from
stable regions (above) to convecting regions (below) occur, as in the Sun.  We will also determine
the effects of the \emph{extent} of the ionizing layer, to determine if this changes the average length
scale of convective elements, and to determine if there is any correlation between the length scale of convective
elements driven at the surface, the atmospheric stratification, and the average power spectrum of motions near the
surface.

\section{Numerical Tools and Feasibility}
I will use the open-source Dedalus\footnote{\url{http://dedalus-project.org/}} pseudospectral framework 
\citep{burns&all2016} to carry out my simulations.  
Dedalus is a flexible solver of partial differential equations,
making it extremely easy to study diverse sets of equations under many different atmospheric
constraints.  I have already published one paper using this tool \citep{anders&brown2017},
will soon submit another paper, and am now adept at using it to create suites of simulations
in short timeframes. Our run scripts for using Dedalus to study stratified atmospheres
are themselves publically available\footnote{\url{https://bitbucket.org/exoweather/polytrope}}.
Blah Blah Blah Reproducibility Blah Blah.

I will primarily study 2D convective solutions in plane-parallel atmospheres in order to gain
intuition about the mean behavior of vertical profiles within the atmosphere.  Once I have a grasp
on how my measurements vary in 2D across parameter space, I will run select 3D simulations to
verify whether or not that behavior holds in 3D, as I did in my previous paper \citep{anders&brown2017}.
In cases where 2D and 3D diverge, I will quantify how and why they do so.
By primarily studying in 2D, and by carefully
selecting my 3D runs once I know which parameters I must examine more carefully, I can complete
a full suite of simulations, such as those in my previous paper \citep{anders&brown2017}, using
roughly 10 million CPU-hours.  Through my advisor, I have access to an allocation on NASA Pleiades
of roughly 20 million CPU-hours/year, so the proposed projects are feasible.



\section{Timeline of proposed work}
\textbf{Year 1 (Fall 2018 - Summer 2019):}
\begin{itemize}
\vspace{-0.2cm}
\item \emph{Project 1:} Conduct literature review on convection with Kramers' opacity early fall 2018.  
Understand past work done, and implement fully compressible equations with Kramers'
opacity in simple atmospheres.  Understand how to control the Mach number in these
atmospheres by end of year 2018.  Run simulations, analyze data, and submit a paper to The Astrophysical Journal
on the nature of convection with Kramers' opacity at both low and high Mach number by
end of spring 2018.
\vspace{-0.2cm}
\item \emph{Project 2:} Conduct literature review on past work done on ionizing convection and moist convection.
Construct appropriate atmospheres for studying ionizing convection, and learn what aspects of
these atmospheres control different aspects of the evolved solutions.
\end{itemize}

\textbf{Year 2 (Fall 2019 - Spring 2020):}
\begin{itemize}
\vspace{-0.2cm}
\item  \emph{Project 2:} Run simulations of ionizing convection, analyze data, and submit a paper to The
Astrophysical Journal by the end of year 2019.
\vspace{-0.2cm}
\item \emph{Academic progression:} Combine work from five published (or submitted) papers into a thesis, to be defended at the end of 
Spring 2020.
\end{itemize}

\section{Relevance to NASA} 
The proposed work fits with NASA's 2014 Strategic Plan objective
1.4:
``Understand the Sun and its interactions with Earth and the solar
system, including space weather.''  Specifically, I aim to help answer
the fundamental question, ``What causes the Sun to vary?'' by understanding
the nature of stratified convection present at the solar photosphere.
This work also aims to answer one of the three overarching science goals
in chapter 4.1 of NASA's 2014 Science Plan: 
``Develop the
knowledge and capability to detect and predict extreme conditions in space to
protect life and society and to safeguard human and robotic explorers beyond
earth.'' In order to understand how to predict space weather appropriately, we
need to understand the processes that cause this weather.  It is clear from
recent work that our fundamental understanding of solar convection is flawed, and
now is an exciting time to clarify our theory and determine which
parts of it fail under closer examination.  

The work has been motivated by data from the Helioseismic and Magnetic Imager (HMI) onboard
the NASA Solar Dynamics Observtory (SDO) spacecraft 
\citep{hanasoge&all2012, greer&all2015, hathaway&all2015}, and will continue to be informed by
new helioiseismic measurements made from SDO data. This work will additionally be informed
by the new measurements which will
be made possible by the upcoming joint NASA-ESA Solar Orbiter's Polarimetric and 
Helioseismic Imager (PHI), and will help explain conundrums arising from those observations.


\section{Summary}
Recent observations call into question our fundamental understanding of stratified
convection in systems such as the solar convection zone \citep{hanasoge&all2012, greer&all2015}.
We propose two focused, scoped studies of the mechanisms which drive the Sun's convection at
the base and top of the solar convection zone. 
These studies will carefully probe the specific physics of these mechanisms and compare the
nature of convection \emph{with} these elements to simpler studies \emph{without} them.
Due to the developed nature of
our computational tool, Dedalus, the simulations for these projects can be implemented and
carried out on short timescales, and the body of work proposed here will be finished within
two years.


\newpage
\bibliographystyle{apj}
\bibliography{biblio}
\end{document}
