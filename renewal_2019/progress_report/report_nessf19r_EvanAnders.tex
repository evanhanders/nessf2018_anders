\documentclass[aasms,12pt]{article}
\usepackage{natbib}
\setlength{\bibsep}{0pt plus 0.3ex}
\usepackage[margin=1in]{geometry}
\usepackage{sectsty}
\usepackage{graphicx}
\usepackage{hyperref}
\usepackage{epstopdf}
\usepackage[skip=2pt,font=small]{caption}
\captionsetup{width=\textwidth}
\usepackage{amssymb, amsmath, amsfonts, xcolor}
\hypersetup{
    colorlinks,
    linkcolor={red!50!black},
    citecolor={blue!80!black},
    urlcolor={blue!80!black}
}


\sectionfont{\normalsize}
\subsectionfont{\small}


%\citestyle{aa}
\newcommand{\sol}{\ensuremath{\odot}}
\newcommand{\RB}{Rayleigh-B\'{e}nard }
\newcommand{\grad}{\ensuremath{\nabla}}

\usepackage{fancyhdr}
\pagestyle{fancy}
\fancyhf{} % sets both header and footer to nothing
\renewcommand{\headrulewidth}{0pt}
\cfoot{\footnotesize{\thepage}}





\begin{document}
\begin{center}
%   \large\textbf{Towards a solution to the solar convective conundrum:}\\
   \large\textbf{Fundamental studies into the solar convective conundrum:}\\
   \large\textbf{Do giant cells exist?}\\
   \vspace{0.2cm}
   \large{Evan H. Anders}\\
   \vspace{0.2cm}
   \normalsize\textit{Advisor: Benjamin P. Brown}\\
   \normalsize\textit{Laboratory for Atmospheric and Space Physics (LASP) \& University of Colorado -- Boulder}\\
\end{center}

\vspace{-0.6cm}
\section{Background \& Motivation}
The Sun exhibits a 22-year active magnetic cycle.
An organized dynamo seated in the turbulent plasma
motions of the solar convection zone drives this magnetism.
Solar magnetism manifests itself in the collection of phenomena generally
referred to as solar activity, including magnetic storms and coronal mass
ejections.  Energetic particles ejected by this activity propagate towards Earth, 
threatening to disrupt power grids and aircraft operations in addition to endangering astronauts and satellites.
Understanding the nature of the dynamo that generates the Sun's magnetic fields is a critical
step to protecting our society from the threats of solar activity \citep{charbonneau2014}.

The solar dynamo is powered by convection, and
an understanding of that convection is essential to comprehending
how solar magnetism is generated. The early work of \cite{graham1975}, \cite{hurlburt&all1984},
and others who studied convection in plane-parallel atmospheres
provided rich insight into the nature of solar-like stratified convection.  From this basis,
the field has blossomed into one which now regularly creates complex, 3D global models of convectively-driven
dynamos (e.g., \cite{brown&all2010} and \cite{guerrero&all2016})
and smaller scale local area models with more complex physics 
(e.g., \cite{stein&nordlund2012} and \cite{rempel2014}).
These efforts have taught us a great deal about the nature of convection, and
beautiful modern simulations even visually resemble the convection observed on the solar surface.

Unfortunately, the great advances made in computational prowess within the solar convection
community have surpassed our fundamental knowledge in the field. This is clear in the
``Solar Convective Conundrum,'' in which observations and theory starkly disagree.
The helioseismic measurements of \cite{hanasoge&all2012} and \cite{greer&all2015} (Fig. \ref{fig:fig1}a)
showed an absence of power in the solar velocity spectrum at large length scales.  Simulations
hypothesized that large-scale
``giant cells'' should be driven by deep convective motions and visible throughout the solar convective
zone, but we do not see these giant cells.
Even simpler Doppler measurements of the velocity fields at the solar surface, which are not muddied
by complex helioseismic inversions, lack
giant cells (\citealt{hathaway&all2015} \& Fig. \ref{fig:fig1}b).  
The motions of surface granules and the slightly deeper supergranules are clearly present, but no
larger length scale is distinct.  These combined observational inferences clearly show that the lack of
giant cells remains a conundrum and deserves further exploration.

%\begin{figure}[t]
%\centering
%\includegraphics[width=\textwidth]{figs/fig1.png}
%\caption{(a) Ring-diagram helioseismic observations of the solar velocity power spectrum at various depths.  
%Here, velocity power decreases toward larger scales, unlike what is expected from simulations
%\citep{greer&all2015}.  (b) A spectrum of horizontal
%velocities at the solar surface, obtained using line-of-sight Doppler velocities \citep{hathaway&all2015}.
%The length scales of surface granules and deeper supergranules appear as distinct features, but
%the hypothesized giant cells are not observed at low wavenumber.
%        \label{fig:fig1}}
%\end{figure}

The work of \cite{lord&all2014} exemplifies why the absence of giant cells is disturbing.
By performing radiative magnetohydrodynamic simulations of the solar photosphere and convective zone using the
MURaM code, the authors
showed that the length scale of convective motions is determined by the depth in the atmosphere at 
which they are driven. In general, motions which are driven deeper in the atmosphere should have
larger length scales.
Due to mass conservation in upflows, these deep motions which are driven at high density should imprint
strongly on the surface motions. Thus,
the lack of giant cells in observations implies that current models are missing a piece of physics. 
Either processes in the solar convection zone mask the deeper motions, or \emph{they are never driven
in the first place}.

In order to determine the fate of giant cells, we must return to simple models of convection to
understand how nonlinear convective dynamics influence the underlying atmospheric stratification.
As convection is only driven where the atmosphere is unstably stratified, processes which stabilize the lower convective
zone should prevent the generation of giant cells and affect surface observations.
Here I present two small experiments which aim to elucidate fundamental aspects of solar
convection. I propose to determine whether either of these mechanisms is the reason that we do not see giant cells
at the solar surface. I will use methods similar to those employed in my recently published work
\citep{anders&brown2017} to create comprehensible, controlled experiments studying turbulent
compressible convection in stratified atmospheres 
to gain a deep understanding of the underlying convective physics.
Furthermore, the knowledge gained in forthcoming work (Anders, Brown, \& Oishi 2018, submitted to PRFluids),
in which we discovered a mechanism for self-consistently converging convective simulations on short timescales,
will be used to allow us to probe highly turbulent flows which more closely approximate solar convection on
manageable human timescales.

\section{Proposed Project}
In simulations of convection, motions are often driven by enforced boundary conditions
on the thermodynamic state.  Boundary layers at the top and bottom of the atmosphere naturally arise,
and convection is strongly driven within those boundary layers.  Convective driving in the Sun is more
complex. A positive radial gradient of opacity within the Sun decreases how efficiently radiation carries
the solar luminosity with increasing height in the convection zone.
This results in a divergence of radiative flux which deposits
energy in the convective layers, and this energy must in turn be carried by convective motions.
In other words, the convection in the Sun is not driven by a sharp lower boundary but rather
through naturally occuring internal heating.
Further, the upper boundary layer of the solar convective zone arises because of 
radiative losses at the photosphere paired with the ionization and 
recombination of hydrogen, not because of an imposed wall at the solar photosphere. These
two effects -- the internal driving of solar convection by opacity effects 
and the driving of convection at the solar surface 
by hydrogen ioniziation -- have not been fully explored in recent literature.

The exciting work of \cite{kapyla&all2017}
exhibited atmospheres in which deep layers of the convection
zone are stable, a setup in which giant cells would not be driven.  
The authors attribute these deep, stable layers to their inclusion of Kramers' opacity
effects which drive convection internally in a manner similar to that in the Sun.  
However, forthcoming work by my advisor and collaborators (Brown et al. 2018 in prep, Oishi et al. 2018 in prep) 
shows that 
stable lower-layers of convective zones naturally arise where stable regions lie below convection
zones and convective motions are at low Mach number.
Further, first results from work that I am conducting on stratified, internally
heated convection (modeled after simpler studes, e.g., \cite{goluskin&spiegel2012}),
show that these stable convecting layers arise naturally when internal heating is the mechanism which
drives convection, even when realistic opacities are not used.

Since the process of internal heating -- not the complex form of a Kramers' opacity --
appears to be the fundamental cause of the stratification effects seen recently by \cite{kapyla&all2017},
the first project that I propose in year 1 is a careful study of the effects of Kramers' opacity on convection.
The second project I propose in year 2 is a careful study of the effects of hydrogen ionization near the surface
of the Sun, building on the previous work of e.g., \cite{rast&toomre1993}.

\vspace{-0.25cm}
\subsection{Project 1: Effect of Kramers' opacity on solar convection}
\vspace{-0.15cm}
The transport of heat within an optically thick atmosphere, in the absence of convective transport,
is often quantified by Fourier's law of conduction \citep{lecoanet&all2014}, in which the radiative
flux is proportional to the conductivity and the temperature gradient.  
Many careful studies of convection employ a constant radiative conductivity.
While a constant conductivity in time and space allows for the creation of simple measurements of the
heat transport in the evolved atmosphere, such an assumption about
the conductivity is coupled with unrealistic assumptions regarding the functional form of the opacity.  

A constant radiative
conductivity is the go-to choice for many in the physics community who study incompressible
\RB convection; however, it is often not the choice for those in the heliophysics or astrophysics
communities.  Instead, these communities generally employ a radiative conductivity which is a function
of the temperature, density, and the Kramers' opacity \citep{barekat&brandenburg2014, brandenburg2016, kapyla&all2017}.
As a result, the conductivity varies greatly throughout the depth of the atmosphere, which 
more carefully models natural physics but also makes the solution harder to interpret.

In order to study the importance of Kramers' opacity, a frame of reference must be constructed
in which to study this varying opacity. For example, what atmospheric parameter determines the Mach number
of evolved flows?  At what value of the Rayleigh number
does convection turn on (and thus, at what \emph{supercriticality} are simulations in past studies)?
What are the parameters of the initial state that determine key quantities
of the evolved convection, and what can we learn about the evolved dynamics from them?
Only by answering these simple questions can a careful study of Kramers' opacity be carried out.
Through answering similar questions about basic polytropic systems, I was able to determine that
regardless of Mach number, basic stratified compressible convection \emph{transports heat
in the same manner as unstratified, incompressible \RB convection} \citep{anders&brown2017}.

After determining how to carry out controlled experiments studying the role of Kramers' opacity,
I will study the importance of the nonlinear nature of this opacity on convection.
In downflows (where density is high and temperature is low), the radiative conductivity
should be low compared to upflows.
However, at low Mach number, where fluctuations in temperature and density are small,
we anticipate that this effect will be unimportant.  In the Sun,
the Mach number of convection ranges from nearly Mach 1 at the surface to O($10^{-5}$) in the
deep interior.  Thus, it is important to understand how this
complex form of opacity interacts with convection at both high and low Mach number in order to understand
how it influences solar convection.

In summary, my goal in year 1 is to quantify the importance of the varying opacity felt by
solar convection. I aim to understand the importance of
nonlinearities in the opacity on nonlocal convective transport and atmospheric stratification.  If
comparatively high opacity in the downflows enhances the importance of nonlocal transport at
all Mach numbers, then the ``entropy rain'' addition to convective theory expanded by
\cite{brandenburg2016} and explored by \cite{kapyla&all2017} could be an essential element
of solar convection.  Nonlocal transport mechanisms of this nature 
could drastically change the stratification of deep convection. If marginal stability in the deep 
convection zone is achieved because of nonlocal transport effects associated with this opacity,
it would imply that the nature of Kramers' opacity helps lead to the lack of observed giant cells.

\vspace{-0.25cm}
\subsection{Project 2: Solar convection influenced by hydrogen ionization and recombination}
\vspace{-0.15cm}
Convection is strongly driven at the solar surface in part by the ionization and recombination of hydrogen.
This piece of physics is absent from many studies of solar convection. Instead, surface convection
is often driven by either an imposed entropy draining layer at the upper boundary \citep{kapyla&all2017}
or the natural thermal boundary layer that forms near the upper surface \citep{anders&brown2017}.
These methods have a considerable problem in that the size of low entropy
convective elements which form at the surface are determined either by the pre-imposed size of the
entropy draining region or the natural size of the thermal boundary layer (which depends on
the opacity).

The convective elements driven at the solar photosphere (granules) are large compared to the local
thermal diffusion length scale.  Their size is likely determined by the vertical extent of the local convective
boundary layer, which in turn is partially set by the depth of hydrogen ionization.  The relatively large
scale of granules may drive convective elements whose size helps them remain coherent as they sink through the depth of the solar
convection zone. If cool surface elements reach the base of the convection zone intact, they could drastically
alter the mean stratification of the deep convective zone.  
I aim to study this likely effect of hydrogen ionization on the stratification of convecting atmospheres.
I will use methods which mimic observations, such as examining the surface velocity power spectrum 
(e.g., Fig. \ref{fig:fig1}), to determine if hydrogen ionization modifies the stratification in a manner that
is measurable at the solar photosphere through its impacts on the convective motions.

I will implement the basic nature of hydrogen ionization and recombination through the use of
a nonlinear equation of state built around a single-atomic-level model of particles,
similar to that used in \cite{rast&toomre1993}.
This work will further be guided by prior studies on moist convection \citep{leconte&all2017},
in which phase changes resulting in cloud formation are studied. I will examine the effects of
hydrogen ionization on atmospheric stratification in two ways.  First, I will determine how the
location of the ionizing layer changes the convection; I will study simulations in which the ionizing
height is at various depths within the domain and others in which it is just outside of the domain.
I will also study
how the vertical extent of the region of partial ionization, the region in which hydrogen transitions
from fully neutral to fully ionized, affects atmospheric stratification and convective driving.
I hypothesize that, in the appropriate solar regime,
these two control parameters likely change the driving scale of convective elements within the
ionizing region. 

In summary, in year 2 I will come to understand the role that hydrogen ionization plays in determining
the length scale at which convective elements are driven at the solar photosphere.
If large elements are driven at the surface and are capable of
descending to the bottom of the convective zone, they could very
well restratify the atmosphere and prevent the generation of giant cells.


\vspace{-0.2cm}
\section{Timeline of proposed work}
\vspace{-0.2cm}
\textbf{Year 1 (Fall 2018 - Summer 2019):}
\begin{itemize}
\vspace{-0.2cm}
\item \emph{Project 1:} Conduct literature review on convection with Kramers' opacity early fall 2018.  
Understand past work done, and implement fully compressible equations with Kramers'
opacity in simple atmospheres.  Understand how to control the Mach number in these
atmospheres by end of year 2018.  Run simulations, analyze data, and submit a paper to The Astrophysical Journal
on the nature of convection with Kramers' opacity at both low and high Mach number by
end of spring 2019.
\vspace{-0.2cm}
\item \emph{Project 2:} Conduct literature review on past work done on ionizing convection and moist convection.
Construct appropriate atmospheres for studying ionizing convection, and learn what properties of
these atmospheres control different aspects of the evolved solutions.
\end{itemize}

\vspace{-0.2cm}
\noindent
\textbf{Year 2 (Fall 2019 - Spring 2020):}
\begin{itemize}
\vspace{-0.2cm}
\item  \emph{Project 2:} Run simulations of ionizing convection, analyze data, and submit a paper to The
Astrophysical Journal by the end of year 2019.
\vspace{-0.2cm}
\item \emph{Academic progression:} Combine work from five published (or submitted) papers into a thesis, to be defended at the end of 
Spring 2020.
\end{itemize}

\vspace{-0.2cm}
\section{Numerical Tools and Feasibility}
\vspace{-0.2cm}
I will use the open-source Dedalus\footnote{\url{http://dedalus-project.org/}} pseudospectral framework 
\citep{burns&all2016} to carry out my simulations.  
Dedalus is a flexible solver of partial differential equations,
making it extremely easy to study diverse sets of equations under many different atmospheric
constraints.  I have already published one paper using this tool \citep{anders&brown2017}
and am now adept at using it to create suites of simulations
in short time frames. Our run scripts for using Dedalus to study stratified atmospheres
are themselves publically available\footnote{\url{https://bitbucket.org/exoweather/polytrope}}.
By using an open-source code base and opening our run scripts to the community, we hope to
make our results reproduceable and decrease barriers for future users in 
\emph{extending} our studies towards a more complete understanding of convection.


I will primarily study 2D convective solutions in plane-parallel atmospheres in order to gain
intuition about the mean behavior of vertical profiles within the atmosphere.  Once I have a grasp
on how my measurements vary in 2D across parameter space, I will run select 3D simulations to
verify whether or not that behavior holds in 3D, as I did in my previous paper \citep{anders&brown2017}.
In cases where 2D and 3D diverge, I will quantify how and why they do so.
By primarily studying in 2D, and by carefully
selecting my 3D runs once I know which parameters I must examine more carefully, I can complete
a full suite of simulations, such as those in my previous paper \citep{anders&brown2017}, using
roughly 10 million CPU-hours.  Through my advisor, I have access to an allocation on NASA Pleiades
of roughly 20 million CPU-hours/year, so the proposed projects are feasible.




\vspace{-0.2cm}
\section{Relevance to NASA} 
\vspace{-0.2cm}
The proposed work aligns with NASA's 2014 Strategic Plan objective
1.4:
``Understand the Sun and its interactions with Earth and the solar
system, including space weather.''  Specifically, I aim to help answer
the fundamental question, ``What causes the Sun to vary?'' by understanding
the nature of stratified convection present at the solar photosphere and in the deep interior.
This work also aims to answer one of the three overarching science goals
in chapter 4.1 of NASA's 2014 Science Plan: 
``Develop the
knowledge and capability to detect and predict extreme conditions in space to
protect life and society and to safeguard human and robotic explorers beyond
earth.'' In order to understand how to predict space weather appropriately, we
need to understand the processes that cause this weather.  Recent work
shows that our fundamental understanding of solar convection is flawed, and our theory needs
clarification.

This work has been motivated by data from the Helioseismic and Magnetic Imager (HMI) onboard
the NASA Solar Dynamics Observtory (SDO) spacecraft 
\citep{hanasoge&all2012, greer&all2015, hathaway&all2015}. This work will additionally be informed
by the new measurements which will
be made possible by the upcoming joint NASA-ESA Solar Orbiter's Polarimetric and 
Helioseismic Imager (PHI), and will help explain conundrums arising from those observations.


\vspace{-0.2cm}
\section{Summary}
\vspace{-0.2cm}
Recent observations call into question our fundamental understanding of solar
convection \citep{hanasoge&all2012, greer&all2015, hathaway&all2015}.
We propose two focused, scoped studies of the mechanisms which drive the Sun's convection at
the solar photosphere and deep in the interior.
These studies will carefully probe the effects of hydrogen ionization and Kramers' opacity, comparing
convective simulations with these elements to simpler studies without them.
Due to the developed nature of
our computational tool, Dedalus, and extensive experience using that tool \citep{anders&brown2017},
the simulations for these projects can be implemented and
carried out on short time scales, and the body of work proposed here will be finished within
two years.


\newpage
\bibliographystyle{apj}
\bibliography{biblio}
\end{document}
