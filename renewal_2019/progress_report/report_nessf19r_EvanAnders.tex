\documentclass[aasms,12pt]{article}
\usepackage{natbib}
\setlength{\bibsep}{0pt plus 0.3ex}
\usepackage[margin=1in]{geometry}
\usepackage{sectsty}
\usepackage{graphicx}
\usepackage{hyperref}
\usepackage{epstopdf}
\usepackage[skip=2pt,font=small]{caption}
\captionsetup{width=\textwidth}
\usepackage{amssymb, amsmath, amsfonts, xcolor}
\hypersetup{
    colorlinks,
    linkcolor={red!50!black},
    citecolor={blue!80!black},
    urlcolor={blue!80!black}
}


\sectionfont{\normalsize}
\subsectionfont{\small}


%\citestyle{aa}
\newcommand{\sol}{\ensuremath{\odot}}
\newcommand{\RB}{Rayleigh-B\'{e}nard }
\newcommand{\grad}{\ensuremath{\nabla}}

\usepackage{fancyhdr}
\pagestyle{fancy}
\fancyhf{} % sets both header and footer to nothing
\renewcommand{\headrulewidth}{0pt}
\cfoot{\footnotesize{\thepage}}

\graphicspath{{./}{figs/}}




\begin{document}
\begin{center}
%   \large\textbf{Towards a solution to the solar convective conundrum:}\\
   \large\textbf{Progress report for: ``Fundamental studies into the solar convective conundrum: Do giant cells exist?''}\\
   \vspace{0.2cm}
   \large{Evan H. Anders}\\
   \vspace{0.2cm}
   \normalsize\textit{Advisor: Benjamin P. Brown}\\
   \normalsize\textit{Laboratory for Atmospheric and Space Physics (LASP) \& University of Colorado -- Boulder}\\
\end{center}

\vspace{-0.6cm}
\section{Work accomplished Fall 2018 - Feb 2019:}
My work in Fall 2018-Spring 2019 was instructed by both an unanticipated opportunity which arose over the
summer of 2018 and an unusual development (which has led to another unanticipated opportunity)
in the original research plan. I will describe both 
of these below, including how they are informing the work of spring and summer 2019.

\subsection{Unanticipated opportunity: Predicting the Rossby number in rotating systems}
Preliminary work into rotating convection in our research group revealed some interesting results
during the summer of 2018, and I chose to gain a greater understanding of these results and
publish them during fall 2018. This work was recenly published in 
The Astrophysical Journal \citep{anders&all2019}.

Recent work by e.g., \citet{featherstone&hindman2016} suggests that rotational effects could be
one possible solution or explanation
to the ``Solar Convective Conundrum,'' the apparent absence of large-scale motions at the
solar photosphere and the focus of my NESSF proposal. More specifically, if flows
in the deep solar convective interior are heavily influenced by the rotation of the Sun, they
may not manifest themselves in the way that traditional mixing length theory or simulations
predict. If rotation is the dominant force on these deep convective flows, they are
at very low \emph{Rossby number} (Ro). Unfortunately, Ro is an \emph{output} parameter
in convective simulations and it is difficult to specify its value \emph{a priori}.
In our recently published work, \citet{anders&all2019}, 
we demonstrated a manner of specifying the Rossby number of simulations in the initial
conditions, as we demonstrate in Fig.~\ref{fig:rossby}, which is Figure 1 of 
\citet{anders&all2019}.

This recently published work has lead to fruitful discussions with multiple scientists across
campus at CU Boulder, including those who are experts in rotating convection 
(Drs.~Keith Julien and Ian Grooms)
and also experts in solar convection simulations (Dr. Nick Featherstone).
These discussions have led to potential collaborations which further explore the results
of \citet{anders&all2019}, and I will describe these possibilities
more fully in section \ref{sec:future_rotation}.

\begin{figure}[b!]
\centering
\includegraphics[width=\textwidth]{parameter_space.pdf}
\caption{Figure 1 of \citet{anders&all2019} is shown. In (a), we show the parameter space
of rotating convection, with Rayleigh number (Ra, ratio of buoyant driving to diffusive damping)
on the $y$-axis and the Taylor number (Ta, strength of rotation compared to diffusive damping)
on the $x$-axis. We also show three types of ``paths'' through this space that simulations
can follow in blue (constant supercriticality, $\mathcal{S}$), green
(constant ``convective Rossby number,'' Ro$_{\text{c}}$), and orange 
(constant ``predictive Rossby number,'' Ro$_{\text{p}}$). 
In (b), we show that along the orange paths, at
constant Ro$_\text{p}$, the evolved Rossby number (Ro) of simulations is
roughly constant across orders of magnitude of Ra, while the other two paths show
increasing or decreasing Ro. In (c), we show that the evolved Rossby number
is roughly a function of Ro$_\text{p}$ alone, and not significantly a function
of the commonly-used Ro$_{\text{c}}$. These results imply that by choosing the value of
Ro$_\text{p}$ and walking along these paths in parameter space, the degree of rotational
constraint of simulations can be chosen by construction. If the solar interior is, in fact,
at low Ro, this finding allows us to probe that dynamical regime.
\label{fig:rossby}}
\end{figure}

%\newpage
\subsection{Convection with Kramer's opacity and thermals}
\label{sec:thermals}
During the fall of 2018, I additionally
worked on ``project 1'' in our original NESSF proposal, as quoted here:

\begin{quote}
``Conduct literature review on convection with Kramers' opacity early fall 2018.  
Understand past work done, and implement fully compressible equations with Kramers'
opacity in simple atmospheres.  Understand how to control the Mach number in these
atmospheres by end of year 2018.  Run simulations, analyze data, and submit a paper to The Astrophysical Journal
on the nature of convection with Kramers' opacity at both low and high Mach number by
end of spring 2019.''
\end{quote}

Through this preliminary work, I have realized that the fundamental process that I proposed
to study is that of \emph{entrainment}, and that the initially proposed project is too complex
of a starting point for a detailed study of this phenomenon. 
Recently, Dr. Daniel Lecoanet (Princeton) has performed an in-depth study of entrainment 
in Boussinesq thermals \citep[][in review]{lecoanet&jeevanjee2018}. A thermal is a region of
cold (or hot) fluid that evolves from rest according to buoyant forces, and they were able to
easily study entrainment in these thermals in both the laminar and turbulent regime.
I have formed a collaboration with Dr. Lecoanet in order to extend his work to study 
entrainment of thermals in stratified domains under the fully compressible equations.
These simulations are, in some way, a simple model of a phenomenon that some believe to be
a solution to the Solar Convective Conundrum: ``entropy rain,'' as described by
\citet{brandenburg2016}. These simple thermal simulations will allow us to
understanding how the complexities of the fully compressible equations
and highly stratified domains will be important to understanding whether or not the entropy
rain hypothesis is vald.

\newpage
I have been studying stratified thermals since late fall 2018, and currently have a
fully functioning code for simulating and analyzing thermals, as in Fig.~\ref{fig:thermals}.
We aim to come up with a robust theory of the behavior of laminar thermals 
(Reynolds number [Re] $\sim$ 600, Fig.~\ref{fig:thermals})
which describes the downward propagation and entrainment of these fluid regions 
at both small and large stratification. Once we fully understand this theory, we will
study whether this linear theory describes the evolution of turbulent thermals (Re $\sim$ 6000),
whose evolution more accurately reflects that of the solar context. This work will be
submitted for publication by the end of spring 2019, as laid out in the original NESSF timeline.

\begin{figure}[t]
\centering
\includegraphics[width=\textwidth]{thermal_snapshot.pdf}
\caption{Snapshots of the vertical velocity field in the $y-z$ plane of 3D simulations are shown
at late times in the evolution of laminar thermals (Reynolds number $\sim$ 600).
From left to right, the atmosphere through
which the thermal is falling is stratified by $n_\rho = [0.1, 0.5, 1, 2, 3, 4, 5]$ density
scale heights, so the simulation on the left is essentially in the Boussinesq limit of
\citet{lecoanet&jeevanjee2018} and the simulation on the right is highly stratified. The thermal
tracking algorithm of \citet{lecoanet&jeevanjee2018} has been extended to work robustly in 
stratified domains, and the black outlines show the boundaries of the thermal obtained by
this algorithm. These preliminary simulations are informing work that will be submitted for
publication by the end of spring 2019.
\label{fig:thermals}}
\end{figure}



\section{Future plans for 2019-2020}
\label{sec:future_plans}
Forthcoming work according to the original proposal is as follows:

\begin{quote}
\textbf{Year 1 (Summer 2019):}
\begin{itemize}
\item Conduct literature review on past work done on ionizing convection and moist convection.
Construct appropriate atmospheres for studying ionizing convection, and learn what properties of
these atmospheres control different aspects of the evolved solutions.
\end{itemize}

\vspace{-0.2cm}
\noindent
\textbf{Year 2 (Fall 2019 - Spring/Summer 2020):}
\begin{itemize}
\vspace{-0.2cm}
\item Run simulations of ionizing convection, analyze data, and submit a paper to The
Astrophysical Journal by the end of year 2019.
\vspace{-0.2cm}
\item \emph{Academic progression:} Combine work from five published (or submitted) papers into a thesis, to be defended at the end of 
Spring 2020.
\end{itemize}
\end{quote}

This general outline still holds true to my plans for the upcoming academic year: I will
do research this summer and next fall, then graduate in the spring or early summer and 
move on to a postdoctoral position at a new university.

Currently, my plan is to work on the originally proposed research (modern, parameterized simulations
of ionizing convection) in an effort to gain an understanding of the size of convective elements
driven at the surface. However, due to the unexpected turns which arose over the past year,
there are a few other projects that I may also investigate coincidentally with these
ionizing simulations.  While it would be impossible to study all three of the following
projects coincidentally with ionizing convection, 
it should be feasible to work on the initially
proposed work as well as one of the following opportunities that has arisen in discussions
with colleagues and collaborators over the past year:

\subsection{Possible mini-project 1: Kramer's opacity thermals}
In the study of thermals that I am currently conducting to learn about entrainment in the
solar convective envelope (see section \ref{sec:thermals}), I am using a simple,
constant-thermal-conductivity specification in order to study the simple underlying physics.
Adding in a nonlinear, Kramer's-like opacity to these thermal simulations is numerically
trivial, and
with the tools I have already developed it would be very easy to see if such a nonlinear
opacity affects the magnitude of entrainment in these experiments. Such a
complication is beyond the scope of the current paper on stratified thermals 
(e.g., Fig.~\ref{fig:thermals}), but this
small, letter-length project would probe the same scientific answer as my originally proposed
project 1: \emph{does a nonlinear, solar-like opacity reduce entrainment in cold downflows?}

\subsection{Possible mini-project 2: Continuing rotation work}
\label{sec:future_rotation}
As I mentioned earlier, the work published in \citet{anders&all2019} has potentially
launched future collaborations with, e.g., Drs.~Julien, Grooms, and/or Featherstone.
Working alongside these experts, I could further study the results published in
\citet{anders&all2019} with the goal of determining what physical mechanism is
responsible for setting the Rossby number. Put differently, in our recent work we showed
that, empirically, the Rossby number is fairly constant if you hold the ``predictive
Rossby number'' constant. A follow-up project would aim to
understand the physical mechanism in the Navier-Stokes equations that leads to this
result, and understand if the Rossby number can be predicted in the solar regime.

\subsection{Possible mini-project 3: Accelerated evolution of stratified, overshooting convection}
In my second paper \citep{anders&all2018}, I studied a method of fast-forwarding through the long relaxation of the
atmosphere's thermal structure in a simple, boussinesq convective system. A fellow graduate student
in our research group and I have some promising first results of an extension of this method
to stratified, compressible simulations, and I could follow up on these results without
too much time investment. This method shows promise for saving millions 
of cpu-hours in future work by allowing computer time to be used on studying
``useful'' parts of simulations (in which the thermal structure is relaxed and the convective
flows are no longer rearranging it) rather than transient phases. 

\bibliographystyle{apj_title}
\bibliography{biblio}
\end{document}
